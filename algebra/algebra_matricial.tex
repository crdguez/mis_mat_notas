% Template created by Karol Kozioł (www.karol-koziol.net) for ShareLaTeX

%\documentclass[a4paper,spanish,9pt]{extarticle}
%\usepackage[utf8]{inputenc}
%
%\usepackage[T1]{fontenc}
%\usepackage{verbatim}
%\usepackage{graphicx}
%\usepackage{xcolor}
%\usepackage{pgf,tikz}
%\usepackage{mathrsfs}
%
%\usetikzlibrary{shapes, calc, shapes, arrows, math, babel}
%
%\usepackage{amsmath,amssymb,textcomp}
%\everymath{\displaystyle}
%
%\usepackage{times}
%\renewcommand\familydefault{\sfdefault}
%\usepackage{tgheros}
%\usepackage[defaultmono,scale=0.85]{droidmono}
%
%\usepackage{multicol}
%\setlength{\columnseprule}{0pt}
%\setlength{\columnsep}{20.0pt}
%
%\usepackage[utf8]{inputenc}
%\usepackage[spanish]{babel}
%\usepackage{eurosym}
%
%\usepackage{graphicx}
%\graphicspath{{../img/}}
%\usepackage{svg}
%
%\usepackage{hyperref}
%
%\usepackage{geometry}
%\geometry{
%a4paper,
%total={210mm,297mm},
%left=10mm,right=10mm,top=10mm,bottom=15mm}
%
%\linespread{1.3}
%
%\newcommand{\samedir}{\mathbin{\!/\mkern-5mu/\!}}
%
%% custom title
%\makeatletter
%\renewcommand*{\maketitle}{%
%\noindent
%\begin{minipage}{0.6\textwidth}
%\begin{tikzpicture}
%\node[rectangle,rounded corners=6pt,inner sep=10pt,fill=blue!50!black,text width= 0.95\textwidth] {\color{white}\Huge \@title};
%\end{tikzpicture}
%\end{minipage}
%\hfill
%\begin{minipage}{0.35\textwidth}
%\begin{tikzpicture}
%\node[rectangle,rounded corners=3pt,inner sep=10pt,draw=blue!50!black,text width= 0.95\textwidth] {\begin{tabular}{cc} \multirow{2}{1cm}{\includegraphics[width=0.15\columnwidth]{header_right}}& \@author \\ & \ies \end{tabular}};
%\end{tikzpicture}
%\end{minipage}
%\bigskip\bigskip
%}%
%\makeatother
%
%% custom section
%\usepackage[explicit]{titlesec}
%\newcommand*\sectionlabel{}
%\titleformat{\section}
%  {\gdef\sectionlabel{}
%   \normalfont\sffamily\Large\bfseries\scshape}
%  {\gdef\sectionlabel{\thesection\ }}{0pt}
%  {
%\noindent
%\begin{tikzpicture}
%\node[rectangle,rounded corners=3pt,inner sep=4pt,fill=blue!50!black,text width= 0.95\columnwidth] {\color{white}\sectionlabel#1};
%\end{tikzpicture}
%  }
%\titlespacing*{\section}{0pt}{15pt}{10pt}
%
%
%% custom footer
%\usepackage{fancyhdr}
%\makeatletter
%\pagestyle{fancy}
%\fancyhead{}
%\fancyfoot[C]{\footnotesize \@author \ - \ies}
%\renewcommand{\headrulewidth}{0pt}
%\renewcommand{\footrulewidth}{0pt}
%\makeatother
%\usepackage{multirow} % para las tablas
%
%
%\title{Funciones}
%\author{Departamento de Matemáticas}
%\date{2014}
%\newcommand{\ies}{IES Pedro Cerrada}
%
%
%
%\begin{document}
%
%\maketitle
%
%
%
%\begin{multicols*}{2}

\section{Matrices}
\section{Determinantes }
\section{Sistemas de ecuaciones}

Dado un sistema de ecuaciones: 

$${\displaystyle \left\{{\begin{matrix}a_{1,1}x_{1}+a_{1,2}x_{2}+\cdots +a_{1,n}x_{n}&=&b_{1}\\a_{2,1}x_{1}+a_{2,2}x_{2}+\cdots +a_{2,n}x_{n}&=&b_{2}\\\vdots & \vdots & \vdots \\a_{m,1}x_{1}+a_{m,2}x_{2}+\cdots +a_{m,n}x_{n}&=&;b_{m}\end{matrix}}\right.}$$

Podemos construir dos matrices:
$${\displaystyle A={\begin{pmatrix}a_{1,1}&\cdots &a_{1,n}\\\vdots &\ddots &\vdots \\a_{m,1}&\cdots &a_{m,n}\end{pmatrix}}}$$
$${\displaystyle A^{*}=\left({\begin{matrix}a_{1,1}&\cdots &a_{1,n}\\\vdots &\ddots &\vdots \\a_{m,1}&\cdots &a_{m,n}\end{matrix}}\right|\left.{\begin{matrix}b_{1}\\\vdots \\b_{m}\end{matrix}}\right)}$$
A las que llamaremos \textbf{matriz de coeficientes} y \textbf{matriz ampliada} respectivamente.

Si llamamos:
$${\displaystyle b={\begin{pmatrix}b_{1}\\\vdots \\b_{m}\end{pmatrix}}}$$ y $${\displaystyle x={\begin{pmatrix}x_{1}\\\vdots \\x_{m}\end{pmatrix}}}$$
Podemos escribir el sistema en notación matricial:

$$A \cdot x = b$$

\subsection{Teorema de Rouché-Frobenius}

Podemos determinar cuántas soluciones tiene un sistema de $n$ ecuaciones lineales a partir del rango de $A$ y $A^*$
\begin{itemize}
    \item Si $rg(A)=rg(A^*) \Rightarrow$ El sistema es compatible
    \begin{itemize}
        \item Si $rg(A)=rg(A^*)=n \Rightarrow$ El sistema es \textbf{compatible determinado}
        \item Si $rg(A)=rg(A^*)=k<m \Rightarrow$ El sistema es \textbf{compatible indeterminado}
        
    \end{itemize}
    \item Si $rg(A)\neq rg(A^*) \Rightarrow$ El sistema es \textbf{incompatible}
\end{itemize}
 
\paragraph{Ejemplo:}
Discutir, y si es posible resolver:
$$\left\{ \begin{matrix}5 x - 2 y + z = -1 \\ - 2 x + y - 3 z = 4 \\ 3 x - y - 2 z = 0 \\ \end{matrix}\right.$$
\subparagraph{Solución:} Sistema incompatible
\begin{itemize}
    \item $rango(A) =2$: \\  $A=\left(\begin{matrix}5 & -2 & 1\\-2 & 1 & -3\\3 & -1 & -2\end{matrix}\right) \begin{array}{c}
     \thicksim   \\
     f_3:f_3-f_2-f_1 \\
 \end{array}
 \left(\begin{matrix}5 & -2 & 1\\-2 & 1 & -3\\0 & 0 & 0\end{matrix}\right)$ \\ y $\left|\begin{matrix}5 & -2 \\-2 & 1 \\ \end{matrix}\right| = 5-4=1\neq0 $ 
    \item $rango(A^*) =3$: \\
    $A^*=\left(\begin{matrix}5 & -2 & 1 & -1\\-2 & 1 & -3 & 4\\3 & -1 & -2 & 0\end{matrix}\right)\begin{array}{c}
     \thicksim   \\
    \end{array}
    \left(\begin{matrix}5 & -2 & 1 & -1\\-2 & 1 & -3 & 4\\0 & 0 & 0 & -3\end{matrix}\right)$ y  
    $\left|\begin{matrix}-2 & 1 & -1 \\1 & 3 & 4 \\ 0&0&-3 \end{matrix}\right| = -3\cdot   
    \left|\begin{matrix}-2 & 1\\1 & -3 \\ \end{matrix}\right|=-3\cdot 5=-15\neq0$
    
    \item $rg(A)=2 \neq rg(A^*)=3 \xRightarrow{\text{R-F}} S.I. $
\end{itemize}

\subparagraph{Observación:}

Si hacemos transformaciones por Gauss llegaríamos a:

$A^*=\left(\begin{matrix}5 & -2 & 1 & -1\\-2 & 1 & -3 & 4\\3 & -1 & -2 & 0\end{matrix}\right)\begin{array}{c}
     \thicksim   \\
     
    \end{array}
    \left(\begin{matrix}5 & -2 & 1 & -1\\0 & \frac{1}{5} & - \frac{13}{5} & \frac{18}{5}\\0 & 0 & 0 & -3\end{matrix}\right)$
    
    donde la última fila representa la ecuación $0\cdot z = -3$ que no tiene solución, 

 \paragraph{Ejemplo}
 
Discutir, y si es posible resolver:
$$\left\{ \begin{matrix}y + z = 1 \\ x + y - 3 z = 8 \\ x - 2 y = -1 \\ \end{matrix}\right.$$
\subparagraph{Discusión:} Sistema compatible determinado
\begin{itemize}
    \item $rango(A) =3$: \\  $A=\left(\begin{matrix}0 & 1 & 1\\1 & 1 & -3\\1 & -2 & 0\end{matrix}\right)$ \\ y $\left|\begin{matrix}0 & 1 & 1\\1 & 1 & -3\\1 & -2 & 0\end{matrix}\right| = 0-2-3-1-0-0=-6\neq0 $ 
    \item $rango(A^*) =3$ ya que $\left|A\right|$ es un menor de orden 3 de $A^*$ no nulo. 
    
    \item $rg(A)=3=rg(A^*)=3 \xRightarrow{\text{R-F}} S.C.D. $
\end{itemize}

\subparagraph{Resolución:} Puesto que el rango es máximo $\Rightarrow \exists A^{-1}$, y por lo tanto podemos resolver el sistema por Gauss, por el método de la matriz inversa, o por la regla de Cramer:
\begin{itemize}
    \item Por Gauss:
    Haciendo las transformaciones $f_1 \rightleftharpoons f_2 , \ f_3\leftharpoondown f_3-f_1, \  f_3\leftharpoondown f_3+3f_2$:\\
    $A^*=\left(\begin{matrix}0 & 1 & 1 & 1\\1 & 1 & -3 & 8\\1 & -2 & 0 & -1\end{matrix}\right)\thicksim
    \left(\begin{matrix}1 & 1 & -3 & 8\\0 & 1 & 1 & 1\\0 & 0 & 6 & -6\end{matrix}\right)$
    \begin{itemize}
        \item Fila 3: $6z = -6 \to z = -1$
        \item Fila 2: $y+z=1 \to y = 1+1 \to y=2$
        \item Fila 1: $x+y-3z=8 \to x=8-2-3 \to x=3$
    \end{itemize}
    \item Por el método del matriz inversa: $A\cdot x=b \Rightarrow x=A^{-1}\cdot b$
    
        \begin{itemize}
            \item $Adj(A^t)=\left(\begin{matrix}-6 & -2 & -4\\-3 & -1 & 1\\-3 & 1 & -1\end{matrix}\right)$
            \item $det(A)=\left|\begin{matrix}0 & 1 & 1\\1 & 1 & -3\\1 & -2 & 0\end{matrix}\right|=-6$
            \item $A^{-1}=\frac{\left(\begin{matrix}-6 & -2 & -4\\-3 & -1 & 1\\-3 & 1 & -1\end{matrix}\right)}{-6}=\left(\begin{matrix}1 & \frac{1}{3} & \frac{2}{3}\\\frac{1}{2} & \frac{1}{6} & - \frac{1}{6}\\\frac{1}{2} & - \frac{1}{6} & \frac{1}{6}\end{matrix}\right)$
        \end{itemize}
        Operando tenemos que: \\
        $\left(\begin{matrix}x\\y\\z\end{matrix}\right)=\left(\begin{matrix}1 & \frac{1}{3} & \frac{2}{3}\\\frac{1}{2} & \frac{1}{6} & - \frac{1}{6}\\\frac{1}{2} & - \frac{1}{6} & \frac{1}{6}\end{matrix}\right) \cdot \left(\begin{matrix}1\\8\\-1\end{matrix}\right)=\left(\begin{matrix}3\\2\\-1\end{matrix}\right)$
    \item Por Cramer: 
    \begin{itemize}
        \item $x=\frac{\left|\begin{matrix}1 & 1 & 1\\8 & 1 & -3\\-1 & -2 & 0\end{matrix}\right|}{-6}=\frac{-18}{-6}=3$
        \item $y=\frac{\left|\begin{matrix}0 & 1 & 1\\1 & 8 & -3\\1 & -1 & 0\end{matrix}\right|}{-6}=\frac{-12}{-6}=2$
        \item $z=\frac{\left|\begin{matrix}0 & 1 & 1\\1 & 1 & 8\\1 & -2 & -1\end{matrix}\right|}{-6}=\frac{6}{-6}=-1$
    \end{itemize}
\end{itemize}

\paragraph{Ejemplo:} Discutir, y si es posible resolver:
$\left\{ \begin{matrix}2 x - y + 3 z = -3 \\ x + y + z = 0 \\ 2 x + 5 y + z = 3 \\ \end{matrix}\right.$

\subparagraph{Discusión:} Sistema compatible indeterminado
\begin{itemize}
    \item $rango(A) =2$: \\ Transformamos a una matriz escalonada \\
    $A=\left(\begin{matrix}2 & -1 & 3\\1 & 1 & 1\\2 & 5 & 1\end{matrix}\right) \begin{array}{c}
     \thicksim   \\
     \end{array}
 \left(\begin{matrix}2 & -1 & 3\\0 & \frac{3}{2} & - \frac{1}{2}\\0 & 0 & 0\end{matrix}\right) $, luego hay dos filas distintas de todo ceros
    \item $rango(A^*) =2$: \\
    $A^*=\left(\begin{matrix}2 & -1 & 3 & -3\\1 & 1 & 1 & 0\\2 & 5 & 1 & 3\end{matrix}\right)\begin{array}{c}
     \thicksim   \\
    \end{array}
    \left(\begin{matrix}2 & -1 & 3 & -3\\0 & \frac{3}{2} & - \frac{1}{2} & \frac{3}{2}\\0 & 0 & 0 & 0\end{matrix}\right)$, de nuevo hay dos filas con elementos no nulos 
    \item $rg(A)=2 = rg(A^*)=2 < 3 \xRightarrow{\text{R-F}} S.C.I. $
\end{itemize}

\subparagraph{Solución:}
Al ser de rango 2 < 3 $\Rightarrow \nexists A^{-1} \to$ No se puede resolver ni por Cramer ni por matriz inversa.
Por Gauss se llega a la matriz escalonada: \\ $A^* \thicksim \left(\begin{matrix}2 & -1 & 3 & -3\\0 & \frac{3}{2} & - \frac{1}{2} & \frac{3}{2}\\0 & 0 & 0 & 0\end{matrix}\right)$:

\begin{itemize}
    \item Fila 3: $0z=0 \to \infty  \ soluciones \to z = \lambda$
    \item Fila 2: $3y-z=3 \to y= 1+\frac{\lambda}{3}$
    \item Fila 1: $2x-y+3z=-3 \to x= -1 - \frac{4\lambda}{3} $
\end{itemize}

\subsection{Ejemplo:}
Discutir y resolver el siguiente sistema con parámetro $k$: \\ $$\left\{ \begin{matrix}k x + k z + y \left(k^{2} + 1\right) = k \\ k y + x + z = 0 \\ k^{2} z + x + y \left(k + 1\right) = 2 k - 1 \\ \end{matrix}\right.$$ 

\textbf{Discusión y resolución por Gauss:} Escalonando la matriz ampliada tenemos\\$A^*= \left(\begin{matrix}k & k^{2} + 1 & k & k\\1 & k & 1 & 0\\1 & k + 1 & k^{2} & 2 k - 1\end{matrix}\right) \thicksim \left(\begin{matrix}1 & k & 1 & 0\\0 & 1 & 0 & k\\0 & 0 & k^{2} - 1 & k - 1\end{matrix}\right)$. \\  De los valores de la última fila podemos concluir:\begin{itemize}\item Si $k = -1 \to$ $$\left(\begin{matrix}1 & -1 & 1 & 0\\0 & 1 & 0 & -1\\0 & 0 & 0 & -2\end{matrix}\right)$$ La última fila es $0z=-2 \to $ S.I.\item Si $k = 1 \to$ $$\left(\begin{matrix}1 & 1 & 1 & 0\\0 & 1 & 0 & 1\\0 & 0 & 0 & 0\end{matrix}\right)$$ La última fila es $0z=0 \to $ S.C.I\begin{itemize}\item $\left(\begin{matrix}0 & 0 & 0 & 0\end{matrix}\right) \to z = \lambda$\end{itemize}\begin{itemize}\item $\left(\begin{matrix}0 & 1 & 0 & 1\end{matrix}\right) \to y = 1$\end{itemize}\begin{itemize}\item $\left(\begin{matrix}1 & 1 & 1 & 0\end{matrix}\right) \to x = -\lambda - 1$\end{itemize}\item si $k\neq [-1, 1]  \to $ S.C.D.\begin{itemize}\item $\left(\begin{matrix}0 & 0 & k^{2} - 1 & k - 1\end{matrix}\right) \to z = \frac{1}{k + 1}$\end{itemize}\begin{itemize}\item $\left(\begin{matrix}0 & 1 & 0 & k\end{matrix}\right) \to y = k$\end{itemize}\begin{itemize}\item $\left(\begin{matrix}1 & k & 1 & 0\end{matrix}\right) \to x = - \frac{k^{3} + k^{2} + 1}{k + 1}$\end{itemize}\end{itemize}  \textbf{Por rangos y determinantes:} \\$\left|A\right|=k^{4} - k^{2} \left(k^{2} + 1\right) + 1 \to \left|A\right|=0 \quad si \quad k = \left [ -1, \quad 1\right ]$\begin{itemize}\item Si $k=-1 \to rg(A)=2 \land rg(A^*)=3 \to $ S.I.\item Si $k=1 \to rg(A)=2 \land rg(A^*)=2 \to $ S.C.I. $\to$ solo se puede resolver por Gauss, ver más arriba\item Si $k=1 \to rg(A)=2 \land rg(A^*)=2 \to $ S.C.D. $\to$ solo se puede resolver por Gauss, (ver más arriba) \\ Por Cramer: \begin{itemize}\item $x=\frac{\left[\begin{matrix}k & k^{2} + 1 & k & k\\0 & k & 1 & 1\\2 k - 1 & k + 1 & k^{2} & 1\end{matrix}\right]}{- k^{2} + 1}=\frac{k^{4} - k^{2} + k - 1}{- k^{2} + 1}=- \frac{k^{3} + k^{2} + 1}{k + 1}$\item $y=\frac{\left[\begin{matrix}k & k & k & k^{2} + 1\\1 & 0 & 1 & k\\1 & 2 k - 1 & k^{2} & k + 1\end{matrix}\right]}{- k^{2} + 1}=\frac{- k^{3} + k}{- k^{2} + 1}=k$\item $z=\frac{\left[\begin{matrix}k & k^{2} + 1 & k & k\\1 & k & 0 & 1\\1 & k + 1 & 2 k - 1 & k^{2}\end{matrix}\right]}{- k^{2} + 1}=\frac{- k + 1}{- k^{2} + 1}=\frac{1}{k + 1}$\end{itemize}\end{itemize}










 














%\end{multicols*}
%
%\end{document}
