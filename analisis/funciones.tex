% Template created by Karol Kozioł (www.karol-koziol.net) for ShareLaTeX

%\documentclass[a4paper,spanish,9pt]{extarticle}
%\usepackage[utf8]{inputenc}
%
%\usepackage[T1]{fontenc}
%\usepackage{verbatim}
%\usepackage{graphicx}
%\usepackage{xcolor}
%\usepackage{pgf,tikz}
%\usepackage{mathrsfs}
%
%\usetikzlibrary{shapes, calc, shapes, arrows, math, babel}
%
%\usepackage{amsmath,amssymb,textcomp}
%\everymath{\displaystyle}
%
%\usepackage{times}
%\renewcommand\familydefault{\sfdefault}
%\usepackage{tgheros}
%\usepackage[defaultmono,scale=0.85]{droidmono}
%
%\usepackage{multicol}
%\setlength{\columnseprule}{0pt}
%\setlength{\columnsep}{20.0pt}
%
%\usepackage[utf8]{inputenc}
%\usepackage[spanish]{babel}
%\usepackage{eurosym}
%
%\usepackage{graphicx}
%\graphicspath{{../img/}}
%\usepackage{svg}
%
%\usepackage{hyperref}
%
%\usepackage{geometry}
%\geometry{
%a4paper,
%total={210mm,297mm},
%left=10mm,right=10mm,top=10mm,bottom=15mm}
%
%\linespread{1.3}
%
%\newcommand{\samedir}{\mathbin{\!/\mkern-5mu/\!}}
%
%% custom title
%\makeatletter
%\renewcommand*{\maketitle}{%
%\noindent
%\begin{minipage}{0.6\textwidth}
%\begin{tikzpicture}
%\node[rectangle,rounded corners=6pt,inner sep=10pt,fill=blue!50!black,text width= 0.95\textwidth] {\color{white}\Huge \@title};
%\end{tikzpicture}
%\end{minipage}
%\hfill
%\begin{minipage}{0.35\textwidth}
%\begin{tikzpicture}
%\node[rectangle,rounded corners=3pt,inner sep=10pt,draw=blue!50!black,text width= 0.95\textwidth] {\begin{tabular}{cc} \multirow{2}{1cm}{\includegraphics[width=0.15\columnwidth]{header_right}}& \@author \\ & \ies \end{tabular}};
%\end{tikzpicture}
%\end{minipage}
%\bigskip\bigskip
%}%
%\makeatother
%
%% custom section
%\usepackage[explicit]{titlesec}
%\newcommand*\sectionlabel{}
%\titleformat{\section}
%  {\gdef\sectionlabel{}
%   \normalfont\sffamily\Large\bfseries\scshape}
%  {\gdef\sectionlabel{\thesection\ }}{0pt}
%  {
%\noindent
%\begin{tikzpicture}
%\node[rectangle,rounded corners=3pt,inner sep=4pt,fill=blue!50!black,text width= 0.95\columnwidth] {\color{white}\sectionlabel#1};
%\end{tikzpicture}
%  }
%\titlespacing*{\section}{0pt}{15pt}{10pt}
%
%
%% custom footer
%\usepackage{fancyhdr}
%\makeatletter
%\pagestyle{fancy}
%\fancyhead{}
%\fancyfoot[C]{\footnotesize \@author \ - \ies}
%\renewcommand{\headrulewidth}{0pt}
%\renewcommand{\footrulewidth}{0pt}
%\makeatother
%\usepackage{multirow} % para las tablas
%
%
%\title{Funciones}
%\author{Departamento de Matemáticas}
%\date{2014}
%\newcommand{\ies}{IES Pedro Cerrada}
%
%
%
%\begin{document}
%
%\maketitle
%
%
%
%\begin{multicols*}{2}

\section{Conjuntos de números reales}
\subsection{Intervalos}
Los intervalos son subconjuntos de la recta real. Cuando el intervalo no está acotado en uno de los extremos decimos que es una semirrecta:

% \adjustbox{max width=\columnwidth}{
% \begin{tabular}{c|c|c|c|c}
%      \hline
%      & NOMBRE & INTERVALO & REPRESENTACIÓN EN LA RECTA REAL & FORMA ALGEBRAICA / SIGNIFICADO\\
%      & & & &
% \end{tabular}
% }
\paragraph{Intervalos Abiertos:}
\begin{itemize}
    \item Dados $a$ y $b \in \mathbf{R}$: Números comprendidos entre $a$ y $b$ sin incluir $a$ y $b$ ó Números mayores que $a$ y menores que $b$
    \item Representación en la recta $\mathbf{R}$ \vfill 
    %\smallskip
    \begin{tikzpicture}[scale=0.6]
\tikzmath{
			\a = -2; \b = 4; 
          }
\draw[latex-latex] (\a - 2,0) -- (\b + 3,0) ;
%\node[below](\a - 2,0){$-\infty$}
\draw[very thick] (\a,0) -- (\b,0);
\path [draw=black, fill=white, very thick] (\b,0) circle (3pt) node[below] {$b$};
\draw [] (\a -2,0) circle (0pt) node [below] {$-\infty$};
\draw [] (\b + 3,0) circle (0pt) node [below] {$+\infty$};

\path [draw=black, fill=white, very thick] (\a,0) circle (3pt) node [below] {$a$};

\end{tikzpicture}
    \item Notación en forma de intervalo: $$\left(a,b\right)$$
    \item Notación en forma de desigualdad: $$\left\{x \in \mathbf{R} | a<x<b \right\} $$

\end{itemize}

\paragraph{Intervalos Cerrados:}
\begin{itemize}
    \item Dados $a$ y $b \in \mathbf{R}$: Números comprendidos entre $a$ y $b$ incluyendo a $a$ y $b$ ó Números mayores o iguales que $a$ y menores o iguales que $b$
    \item Representación en la recta $\mathbf{R}$ \vfill 
    %\smallskip
    \begin{tikzpicture}[scale=0.6]
\tikzmath{
			\a = -2; \b = 4; 
          }
\draw[latex-latex] (\a - 2,0) -- (\b + 3,0) ;
%\node[below](\a - 2,0){$-\infty$}
\draw[very thick] (\a,0) -- (\b,0);
\path [draw=black, fill=black, very thick] (\b,0) circle (3pt) node[below] {$b$};
\draw [] (\a -2,0) circle (0pt) node [below] {$-\infty$};
\draw [] (\b + 3,0) circle (0pt) node [below] {$+\infty$};

\path [draw=black, fill=black, very thick] (\a,0) circle (3pt) node [below] {$a$};

\end{tikzpicture}
    \item Notación en forma de intervalo: $$\left[a,b\right]$$
    \item Notación en forma de desigualdad: $$\left\{x \in \mathbf{R} | a\leq x \leq b \right\} $$

\end{itemize}

\paragraph{Intervalos Semiabiertos:} En los intervalos genéricos anteriores a $a$ y a $b$ se les llama extremos. Si solo se incluye en el conjunto a uno de los extremos se dice que el intervalo es \textbf{semiabierto}.


\paragraph{Semirrectas:} Si uno de los extremos del intervalo es $-\infty$ o $+\infty$, se dice que el conjunto númerico es una \textbf{semirrecta}. La parte correspondiente al infinito siempre se deja abierta.

\paragraph{El conjunto $\mathbf{R}$ en forma de intervalo:} $\mathbf{R}=\left(-\infty,+\infty\right)$

\paragraph{Ejemplos:}
\subparagraph{Intervalo abierto $\left(-1,5\right)$}
\begin{itemize}
    \item Es el conjunto de números comprendidos entre $-1$ y $5$ sin incluir a $-1$ ni a $5$ ó los números mayores que $-1$ y menores que $5$
    \item Representación en la recta $\mathbf{R}$ \vfill 
    %\smallskip
    \begin{tikzpicture}[scale=0.6]
\tikzmath{
			\a = -2; \b = 4; 
          }
\draw[latex-latex] (\a - 2,0) -- (\b + 3,0) ;
%\node[below](\a - 2,0){$-\infty$}
\draw[very thick] (\a,0) -- (\b,0);
\path [draw=black, fill=white, very thick] (\b,0) circle (3pt) node[below] {$5$};
\draw [] (\a -2,0) circle (0pt) node [below] {$-\infty$};
\draw [] (\b + 3,0) circle (0pt) node [below] {$+\infty$};

\path [draw=black, fill=white, very thick] (\a,0) circle (3pt) node [below] {$-1$};

\end{tikzpicture}
    \item Notación en forma de desigualdad: $$\left\{x \in \mathbf{R} | -1<x<5 \right\} $$
\end{itemize}

\subparagraph{Intervalo semiabierto $\left(-1,5\right]$}
\begin{itemize}
    \item Es el conjunto de números comprendidos entre $-1$ y $5$ sin incluir a $-1$ pero sí a $5$ ó los números mayores que $-1$ y menores o iguales que $5$
    \item Representación en la recta $\mathbf{R}$ \vfill 
    %\smallskip
    \begin{tikzpicture}[scale=0.6]
\tikzmath{
			\a = -2; \b = 4; 
          }
\draw[latex-latex] (\a - 2,0) -- (\b + 3,0) ;
%\node[below](\a - 2,0){$-\infty$}
\draw[very thick] (\a,0) -- (\b,0);
\path [draw=black, fill=black, very thick] (\b,0) circle (3pt) node[below] {$5$};
\draw [] (\a -2,0) circle (0pt) node [below] {$-\infty$};
\draw [] (\b + 3,0) circle (0pt) node [below] {$+\infty$};

\path [draw=black, fill=white, very thick] (\a,0) circle (3pt) node [below] {$-1$};

\end{tikzpicture}
    \item Notación en forma de desigualdad: $$\left\{x \in \mathbf{R} | -1<x\leq 5 \right\} $$
\end{itemize}

\subparagraph{Semirrecta $\left(-1,+\infty\right)$}
\begin{itemize}
    \item Es el conjunto de números comprendidos entre $-1$ y $+\infty$ ó los números mayores que $-1$
    \item Representación en la recta $\mathbf{R}$ \vfill 
    %\smallskip
    \begin{tikzpicture}[scale=0.6]
\tikzmath{
			\a = -2; \b = 4; 
          }
\draw[latex-latex] (\a - 2,0) -- (\b + 3,0) ;
%\node[below](\a - 2,0){$-\infty$}
\draw[-latex, very thick] (\a,0) -- (\b+3,0);
%\path [draw=black, fill=white, very thick] (\b,0) circle (3pt) node[below] {$5$};
\draw [] (\a -2,0) circle (0pt) node [below] {$-\infty$};
\draw [] (\b + 3,0) circle (0pt) node [below] {$+\infty$};

\path [draw=black, fill=white, very thick] (\a,0) circle (3pt) node [below] {$-1$};

\end{tikzpicture}
    \item Notación en forma de desigualdad: $$\left\{x \in \mathbf{R} | x>-1 \right\} $$
\end{itemize}

\subsection{Operaciones con conjuntos de números reales}
\paragraph{Unión:}La expresión $A\cup B$ se lee “A unión B” e indica un conjunto que contiene todos los elementos de A y B.
\paragraph{Intersección:}La expresión $A\cap B$ se lee “A intersección B” e indica un conjunto que contiene los elementos comunes de A y B.

\subsection{Repaso de Intervalos, Entornos y Valor Absoluto}

\begin{itemize}
\item Se define el \textbf{valor absoluto} de un número como la distancia al cero. Se calcula tomando el número con signo positivo.

\paragraph*{Ejemplo:} Valor absoluto de -2 
\vfill
\begin{tikzpicture}[scale=0.75]
\tikzmath{
			\a = -2; \b = 4; \aa = -3; \bb = 5 ;
			\aabs = -\a; \babs = \b;
          }

\draw[very thick] (\a,0) -- (\b,0);
\path [draw=black, fill=white] (4,0) circle (2pt);
\path [draw=black, fill=white, thick] (-2,0.0) circle (2pt);
\draw[latex-latex] (\a - 1.5,0) -- (\b + 1.5,0) ;

\foreach \x in  {\aa,...,\bb}
\draw[shift={(\x,0)},color=black] (0pt,3pt) -- (0pt,-3pt);
\foreach \x in  {\aa,...,\bb}
\draw[shift={(\x,0)},color=black] (0pt,0pt) -- (0pt,-3pt) node[below] 
{$\pgfmathprintnumber{\x}$};
  \draw[decorate,decoration={brace}, thick]
    (\a,0.2)--(0,0.2) node[above, midway] 
{$\pgfmathprintnumber{\aabs} = |\pgfmathprintnumber{\a}|$}; 
\end{tikzpicture}

\paragraph*{Distancia entre dos números:} Entre $a$ y $b$ hay una distancia de $|a -b|$ 
\paragraph*{Ejemplo:} Determina la distancia entre los números $-2$ y  $4$:

\begin{tikzpicture}[scale=0.75]

\tikzmath{
			\a = -2; \b = 4; \aa = -3; \bb = 5 ;
			\dist = \b - \a;
          }

\draw[very thick] (\a,0) -- (\b,0);
\path [draw=black, fill=white] (4,0) circle (2pt);
\path [draw=black, fill=white, thick] (-2,0.0) circle (2pt);
\draw[latex-latex] (\a - 1.5,0) -- (\b + 1.5,0) ;

\foreach \x in  {\aa,...,\bb}
\draw[shift={(\x,0)},color=black] (0pt,3pt) -- (0pt,-3pt);
\foreach \x in  {\aa,...,\bb}
\draw[shift={(\x,0)},color=black] (0pt,0pt) -- (0pt,-3pt) node[below] 
{$\pgfmathprintnumber{\x}$};
  \draw[decorate,decoration={brace}, thick]
    (-2,0.2)--(4,0.2) node[above, midway] 
{$\pgfmathprintnumber{\dist} = |(\pgfmathprintnumber{\b}) - (\pgfmathprintnumber{\a})|$}; 
\end{tikzpicture}
\end{itemize}



\section{Funciones}

\subsection{Definición}

En el lenguaje matemático se dice que $y$ es \textbf{función} de $x$ cuando $y$ depende de $x$.

\section{Características de una función}

\subsection{Dominio y recorrido}
El conjunto de los posibles valores de la
variable independiente se llama \textbf{dominio de la función} ($Dom(f)$); y el conjunto de valores que toma la variable dependiente, \textbf{imagen o recorrido de la función} ($Im(f)$).

\paragraph{Ejemplo}
$y=x^2$ o $f(x)=x^2$, $x$ es la variable independiente e $y$ es la variable dependiente. El $Dom(f)=\left\lbrace x \in \mathrm{R} | \exists y=f(x)\right\rbrace$ 	

\begin{tikzpicture}[domain=-3:3,>=triangle 45, scale=0.75]

\tikzmath{
			\a = 1/4; \b = 1; \c = -2; 
			\d = 3;
          }
          
\draw[color=red]    plot (\x,{\a*(\x)^2 + \b*\x + \c})             node[right] {$f(x) =\a x^2$}; 
\draw[very thin,color=gray] (-3.5,\c - 0.5) grid (3.5,\a * 3^2 + \b * 3 + \c - 0.1);
\draw[<->] (-3.5,0) -- (3.9,0) node[right] {$x$};
\draw[<->] (0,\c - 0.5) -- (0,\a * 3^2 + \b * 3 + \c) node[above] {$y$};

\end{tikzpicture}

\subsection{Crecimiento y decrecimiento}
\subsection{Cortes con ejes}
\subsection{Funciones acotadas}
\subsection{Funciones periódicas}
\subsection{Funciones inyectivas y biyectivas}
\subsection{Continuidad}
De manera informal, una función es continua si se puede dibujar con un solo trazo, o lo que es lo mismo, sin levantar "del papel" el "bolígrafo". Para dar una definición


\section{Continuidad y límites}
\subsection{Límites en un punto}
Dado $x_0\neq\pm\infty$, decimos que $\lim_{x \to x_0} f(x) = l $ cuando ocurre que si $x$ toma valores próximos al número $x_0$ (tanto menores como mayores), los correspondientes valores de $f (x)$ se aproximan al número $l$

\subsection{Límites laterales}

\subsubsection{Límites por la derecha} Dado $x_0\neq\pm\infty$, decimos que $\lim_{x \to x_0^+} f(x) = l $ cuando ocurre que si $x$ toma valores próximos al número $x_0$ pero  mayores que él, los correspondientes valores de $f (x)$ se aproximan al número $l$

\subsubsection{Límites por la izquierda} Dado $x_0\neq\pm\infty$, decimos que $\lim_{x \to x_0^-} f(x) = l $ cuando ocurre que si $x$ toma valores próximos al número $x_0$ pero  menores que él, los correspondientes valores de $f (x)$ se aproximan al número $l$

\paragraph{Teorema de unicidad del límite:}
Dado $x_0\neq\pm\infty$,
$$\lim_{x \to x_0} f(x) = l \leftrightarrow \lim_{x \to x_0^-} f(x) = \lim_{x \to x_0^+} f(x)= l $$
Por lo tanto si los límites laterales no coinciden, el límite no existe.

\paragraph{Ejemplo:} Calcula el límite de la función $f(x)=\frac{1}{x}$ cuando $x \to 0$.\\
$$\lim_{x \to 0^-} \frac{1}{x} = -\infty$$



    \begin{tabular}{l l}
      Derivatives and Integrals \\
      \\

      Basic Differentiation Rules \\

      \\
      1. $\frac {d}{dx} [cu] = cu'$ & 2. $\frac{d}{dx} [u \pm v] = u' \pm 
      v'$ \\
      3. $\frac{d}{dx} [uv] = uv' + vu'$ & 4. $\frac{d}{dx} [\frac{u}{v}] 
      = \frac{vu' - uv'}{v^2}$ \\
      5. $\frac{d}{dx} [c] = 0$ & 6. $\frac{d}{dx} [u^n] = nu^{n-1} \quad 
      u'$ \\
      7. $\frac{d}{dx} [x] = 1$ & 8. $\frac{d}{dx} [\mid u \mid] = 
      \frac{u}{\mid u \mid} (u'), \quad u \neq 0$ \\
      9. $\frac{d}{dx} [ln \quad u] = \frac{u'}{u}$ & 10. $\frac{d}{dx} 
      [e^u] = e^u \quad u'$ \\
      11. $\frac{d}{dx} [sin \quad u] = (cos \quad u) u'$ & 12. $\frac{d}
      {dx} [cos \quad u] = -(sin \quad u) u'$ \\
      13. $\frac{d}{dx} [tan \quad u] = (sec^2 \quad u)u'$ & 14. 
      $\frac{d}{dx} [cot \quad u] = -(csc^2 \quad u) u'$ \\
      15. $\frac{d}{dx} [sec \quad u] = (sec \quad u \quad tan \quad u) 
      u'$ & 16. $\frac{d}{dx} [csc \quad u] = -(csc \quad u \quad cot 
      \quad u) u'$ \\
      17. $\frac{d}{dx} [arcsin \quad u] = \frac{u'}{\sqrt{-1 - u^2}}$ & 
      18. $\frac{d}{dx} [arccos \quad u] = \frac{-u'}{\sqrt{1-u^2}}$ \\
      19. $\frac{d}{dx} [arctan \quad u] = \frac{u'}{1 + u^2}$ & 20. 
      $\frac{d}{dx} [arccot \quad u] = \frac{-u'}{1 + u^2}$ \\
      21. $\frac{d}{dx} [arcsec \quad u] = \frac{u'}{\mid u \mid 
      \sqrt{u^2 - 1}}$ & 22. $\frac{d}{dx} [arcsec \quad u] = \frac{-u'}
      {\mid u \mid \sqrt{u^2 - 1}}$ \\
      \\
      Basic Integration Formulas \\
      \\
      1. $\int k \quad f(u) \quad d u = k \int f(u) \quad du$ & 2. $\int 
      \quad [f(u) \pm g (u)] \quad du = \int f(u) \quad du \pm \int g(u) 
      \quad du$ \\
      3. $\int  d u = u + C$ & 4. $\int u^n d u = \frac{u^{n+1}}{n + 1} + 
      C, \quad n \neq -1$ \\
      5. $\int \frac{d}{u} = ln \mid u \mid + C$ & 6. $\int e^u d u = e^u 
      + C$ \\
      7. $\int sin \quad u \quad du = -cos u + C$ & 8. $\int cos \quad u 
      \quad d u = sin \quad u + C$ \\
      9. $\int tan \quad u \quad du = -ln \mid cos \quad u \mid + C$ & 
      10. $\int cot \quad u \quad du = ln \mid sin \quad u \mid + C$ \\
      11. $\int sec \quad u \quad du = ln \mid sec \quad u + tan \quad du 
      \mid + C$ & 12. $\int csc \quad u \quad du = -ln \mid csc \quad u + 
      cot \quad u \mid + C$ \\
      13. $\int sec^2 u \quad du  = tan \quad u + C$ & 14. $\int csc^2 
      \quad u \quad du = -cot \quad u + C$ \\
      15. $\int sec \quad u \quad tan \quad u \quad du = sec \quad u + C$ 
      & 16. $\int csc \quad u \quad cot \quad du = -csc \quad u + C$ \\
      17. $\int \frac{du}{\sqrt{a^2 - u^2}} = arcsin \frac{u}{a} + C$ & 
      18. $\int \frac{du}{a^2 + u^2} = \frac{1}{a} arctan \frac{u}{a} + 
      C$ \\
      19. $\int \frac{du}{u \sqrt{u^2 - a^2}} = \frac{1}{a} arcsec 
      \frac{\mid u \mid}{a} + C$ \\

  \end{tabular}


\usetikzlibrary{arrows,intersections}
\tikzpicture[scale=0.5,
        thick,
        >=stealth',
        dot/.style = {
            draw,
            fill=white,
            circle,
            inner sep=0pt,
            minimum size=4pt
        }
    ]
    \coordinate (O) at (0,0);
    \draw[->] (-0.3,0) -- (8,0) coordinate[label={below:$x$}] (xmax);
    \draw[->] (0,-0.3) -- (0,5) coordinate[label={right:$f(x)$}] (ymax);
    \path[name path=x] (0.3,0.5) -- (6.7,4.7);
    \path[name path=y] plot[smooth] coordinates {(-0.3,2) (2,1.5) (4,2.8) (6,5)};
    \scope[name intersections={of=x and y,name=i}]
        \fill[gray!20] (i-1) -- (i-2 |- i-1) -- (i-2) -- cycle;
        \draw (0.3,0.5) -- (6.7,4.7) node[pos=0.8,below right] {Sekante};
        \draw[red] plot[smooth] coordinates {(-0.3,2) (2,1.5) (4,2.8) (6,5)};
        \draw (i-1) node[dot,label={above:$P$}] (i-1) {} -- node[left] {$f(x_0)$} (i-1 |- O) node[dot,label={below:$x_0$}] {};
        \path (i-2) node[dot,label={above:$Q$}] (i-2) {} -- (i-2 |- i-1) node[dot] (i-12) {};
        \draw (i-12) -- (i-12 |- O) node[dot,label={below:$x_0 + \varepsilon$}] {};
        \draw[blue,<->] (i-2) -- node[right] {$f(x_0 + \varepsilon) - f(x_0)$} (i-12);
        \draw[blue,<->] (i-1) -- node[below] {$\varepsilon$} (i-12);
        \path (i-1 |- O) -- node[below] {$\varepsilon$} (i-2 |- O);
        \draw[gray] (i-2) -- (i-2 -| xmax);
        \draw[gray,<->] ([xshift=-0.5cm]i-2 -| xmax) -- node[fill=white] {$f(x_0 + \varepsilon)$}  ([xshift=-0.5cm]xmax);
    \endscope
\endtikzpicture

\newpage


\begin{tabular}{|c|c|}\hline 
Función  & Derivada  \\ 
\hline
$f(x) = k$  & $f'(x) = 0$   \\ 
$f(x) = x$  &                     $f'(x)= 1$ \\
$f(x) = x^n$ &                    $f'(x)= n\cdot x^{n-1}$\\
$f(x) = \sqrt x $ &      $f'(x)$= $\frac{1}{{2\sqrt x }}$\\
$f(x) = \sqrt[n]{x}$  &     $f'(x)$= $\frac{1}{n\cdot\sqrt[n]{x^{n - 1}}}$\\
$f(x) = \ln{x}$ &                  $f'(x)= \frac{1}{x}$\\
$f(x) = \log_a x$ &              $f'(x)=\frac{1}{x}\cdot\frac{1}{\ln a}$\\
$f(x) = e^x$ &                     $f'(x)=e^x$\\
$f(x) = a^x$ &                     $f'(x)= a^x\cdot\ln a$\\
$f(x) = \sin x$ &               $f'(x)= \cos x$ \\
$f(x) = \cos x$  &              $f'(x)= -\sin x$\\
$f(x) = \tan x$ &                  $f'(x)$= $\frac{1}{{{{\cos }^2}x}}$\\
%f(x) = cotg x              $f'(x)$= $\frac{{ - 1}}{{se{n^2}x}}{\text{ }} = {\text{  -  cose}}{{\text{c}}^{\text{2}}}x$
%f(x) = sec x                $f'(x)$= tg x . sec x
%f(x) = cosec x            $f'(x)$= -cotg x . cosec x
%f(x) = arc sen x        $f'(x)$= $\frac{1}{{\sqrt {1 - {x^2}} }}$
%f(x) = arc cos x         $f'(x)$= $\frac{{ - 1}}{{\sqrt {1 - {x^2}} }}$
%f(x) = arc tg x           $f'(x)$= $\frac{1}{{1 + {x^2}}}$
%f(x) = arc cotg x       $f'(x)$= $\frac{{ - 1}}{{1 + {x^2}}}$
\hline

\end{tabular} 

\begin{tabular}{|c|}\hline 
Operaciones con derivadas  \\ 
\hline
$y=u + v \rightarrow y' = u' + v'$    \\ 
$y=u - v \rightarrow y' = u' - v'$    \\
$y=K\cdot u \rightarrow y' = K \cdot u'$    \\ 
$y=u \cdot v \rightarrow y' = u'\cdot v + u\cdot v'$    \\
$y= \frac{u}{v}  \rightarrow y' = \frac{u'\cdot v - u\cdot v'}{v^2}$    \\  
\hline

\end{tabular} 

\paragraph{Regla de la cadena} Dada una función compuesta: 

$$\left(g \circ f\right)'(x)=g'\left(f(x)\right)\cdot f'\left(x\right) $$  

\subparagraph{Ejemplo}

Dado la función $f(x)$

     




%\end{multicols*}
%
%\end{document}
