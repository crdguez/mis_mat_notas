\documentclass[11pt]{beamer}
\usetheme{Boadilla}
\usepackage[utf8]{inputenc}
\usepackage{amsmath}
\usepackage{amsfonts}
\usepackage{amssymb}

\usepackage{hyperref}
\usepackage{graphicx}
\usepackage[spanish]{babel}

\usepackage{pgf,tikz}

\usetikzlibrary{shapes, calc, shapes, arrows, math, babel, positioning}
\newcommand{\degre}{\ensuremath{^\circ}}
\usepackage{pgf,tikz,pgfplots}
\pgfplotsset{compat=1.15}
\usepackage{mathrsfs}
\usetikzlibrary{arrows}

%\author{}
\title{Probabilidad}
%\setbeamercovered{transparent} 
%\setbeamertemplate{navigation symbols}{} 
%\logo{} 
%\institute{} 
\date{} 
%\subject{} 
\begin{document}

\begin{frame}
\titlepage
\end{frame}

%\begin{frame}
%\tableofcontents
%\end{frame}

\begin{frame}{Experimentos aleatorios}
\begin{block}{}
Un experimento es aleatorio cuando depende de muchos factores y cualquier pequeña modificación de alguno implica obtener un resultado diferente.
\end{block}
\begin{itemize}
 \item \textbf{Aleatorio}: Lanzar un dado y ver el resultado
 \item \textbf{Determinista}: Calcular el tiempo que tarda en caer un objeto al suelo desde una distancia determinada
 \end{itemize} 

\end{frame}



\begin{frame}{Espacio muestral y sucesos}
\begin{itemize}
\item \textbf{Espacio muestral}: Conjunto de los posibles resultados del experimento. Se denota: $E$
\item \textbf{Sucesos simples o elementales}: Cualquiera de los elementos del espacio muestral
\item \textbf{Sucesos compuestos}: Sucesos formados por varios simples. 
\item \textbf{Suceso seguro}: Suceso compuesto por los elementos del Espacio muestral. Se cumple siempre
\item \textbf{Suceso imposible}: Cualquier suceso que no se cumpla nunca. Se denota con el símbolo: $\varnothing$
\item \textbf{Suceso contrario}: Si $A$ es un suceso, $\overline{A}$ es el suceso contrario. Es aquel que se cumple cuando no se cumple $A$
\end{itemize}

\end{frame}

\begin{frame}{Ejemplos:}
Lanzamos un dado y comprobamos la cara que sale:
\begin{itemize}
\item \textbf{Espacio muestral}: $E=\lbrace 1,2,3,4,5,6 \rbrace $
\item \textbf{Sucesos simples o elementales}: $1$, $2$, $3$, $4$, $5$ ó $6$
\item \textbf{Sucesos compuestos}: $A=\lbrace que\ salga\ par\rbrace=\lbrace2,4,6\rbrace$
\item \textbf{Suceso seguro}: $E=\lbrace 1,2,3,4,5,6 \rbrace $
\item \textbf{Suceso imposible}: $\varnothing=\lbrace que\ salga \ mayor \ que \ 6\rbrace$
\item \textbf{Suceso contrario}: Si $A=\lbrace que\ salga\ par\rbrace=\lbrace2,4,6\rbrace$, $\overline{A}=\lbrace que\ salga\ impar\rbrace=\lbrace1,3,5\rbrace$ 
\end{itemize}
\end{frame}


\begin{frame}
{Operaciones con sucesos y relaciones}

\begin{block}{}
la \textbf{unión} de los sucesos $A$ y $B$ es aquel suceso que contiene a todos los elementos de $A$ y a  los de $B$. Se denota: $A\cup B$ 
\end{block}

%\documentclass{standalone}
%\usepackage[utf8]{inputenc}
%\usepackage{graphicx}
%\usepackage{xcolor}
%\usepackage{pgf,tikz}
%\usepackage{mathrsfs}
%\usetikzlibrary{shapes, calc, shapes, arrows, math, babel, positioning}
%
%\begin{document}

   
\begin{tikzpicture}[scale=0.8]
\draw[draw=blue!50, thick] (-2,-2) rectangle (4,2) node[above] {$E$};
    \draw[fill=blue!20, draw=blue!50, thick] 
    	(0,0) circle (1.5cm) node {$A$}
        (0:2cm) circle (1.5cm) node {$B$};
    \node[anchor=south] at (1,1.3) {$A \cup B$};
\end{tikzpicture}

% Original:
%\def\firstcircle{(0,0) circle (1.5cm)}
%\def\secondcircle{(0:2cm) circle (1.5cm)}
%\def\espacio{(-2,-2) rectangle (4,2)}
%
%\colorlet{circle edge}{blue!50}
%\colorlet{circle area}{blue!20}
%
%\tikzset{filled/.style={fill=circle area, draw=circle edge, thick},
%    outline/.style={draw=circle edge, thick}}
%    
%\begin{tikzpicture}[scale=0.8]
%\draw[outline] \espacio node[above] {$E$};
%    \draw[filled] \firstcircle node {$A$}
%                  \secondcircle node {$B$};
%    \node[anchor=south] at (1,1.3) {$A \cup B$};
%\end{tikzpicture}


%\end{document}
\end{frame}

\begin{frame}
{Operaciones con sucesos y relaciones}

\begin{block}{}
la \textbf{intersección} de los sucesos $A$ y $B$ es aquel suceso que contiene a todos los elementos que están tanto en $A$ como en $B$. Se denota: $A\cap B$ 
\end{block}

%\documentclass{standalone}
%\usepackage[utf8]{inputenc}
%\usepackage{graphicx}
%\usepackage{xcolor}
%\usepackage{pgf,tikz}
%\usepackage{mathrsfs}
%\usetikzlibrary{shapes, calc, shapes, arrows, math, babel, positioning}
%
%\begin{document}

\begin{tikzpicture}[scale=0.8]
	\draw[draw=blue!50, thick] (-2,-2) rectangle (4,2) node[above] {$E$};
    \begin{scope}
        \clip (0,0) circle (1.5cm);
        \fill[fill=blue!20, draw=blue!50, thick] (0:2cm) circle (1.5cm);
    \end{scope}
    \draw[draw=blue!50, thick] (0,0) circle (1.5cm) node {$A$};
    \draw[draw=blue!50, thick] (0:2cm) circle (1.5cm) node {$B$};
    %\node[anchor=south] at (current bounding box.north) {$A \cap B$};
    \node[anchor=south] at (1,1.3) {$A \cap B$};
\end{tikzpicture}

   


% Original:
%\def\firstcircle{(0,0) circle (1.5cm)}
%\def\secondcircle{(0:2cm) circle (1.5cm)}
%\def\espacio{(-2,-2) rectangle (4,2)}
%
%\colorlet{circle edge}{blue!50}
%\colorlet{circle area}{blue!20}
%
%\tikzset{filled/.style={fill=circle area, draw=circle edge, thick},
%    outline/.style={draw=circle edge, thick}}
%    
%\begin{tikzpicture}[scale=0.8]
%\draw[outline] \espacio node[above] {$E$};
%    \draw[filled] \firstcircle node {$A$}
%                  \secondcircle node {$B$};
%    \node[anchor=south] at (1,1.3) {$A \cup B$};
%\end{tikzpicture}

%
%\end{document}
\end{frame}

\begin{frame}
{Operaciones con sucesos y relaciones}

 Tomamos como experimento el resultado de lanzar un dado, y los sucesos: \\
\begin{tabular}{l}
$A=\lbrace que\ salga\ par\rbrace=\lbrace2,4,6\rbrace$ \\
$B=\lbrace que\ sea\ mayor\ que\ 3\rbrace=\lbrace4,5,6\rbrace$ \\
$C=\lbrace que\ salga\ impar\rbrace=\lbrace1,3,5\rbrace$
\end{tabular}

\begin{itemize}	
	\item $A\cup B=\lbrace2,4,5,6\rbrace$ \\
	%%\documentclass{standalone}
%\usepackage[utf8]{inputenc}
%\usepackage{graphicx}
%\usepackage{xcolor}
%\usepackage{pgf,tikz}
%\usepackage{mathrsfs}
%\usetikzlibrary{shapes, calc, shapes, arrows, math, babel, positioning}
%
%\begin{document}

\begin{tikzpicture}[scale=0.8]
\draw[draw=blue!50, thick] (-2,-2) rectangle (4,2) node[above] {$E$};
    \draw[fill=blue!20, draw=blue!50, thick] 
    	(0,0) circle (1.5cm) node[left] {$2$}
        (0:2cm) circle (1.5cm) node[right] {$5$};
    \node[anchor=south] at (1,1.3) {$A \cup B$};
    \node[anchor=south] at (1,0.25) {$4$};
   \node[anchor=south] at (1,-0.75) {$6$};

\end{tikzpicture}

   

% Original:
%\def\firstcircle{(0,0) circle (1.5cm)}
%\def\secondcircle{(0:2cm) circle (1.5cm)}
%\def\espacio{(-2,-2) rectangle (4,2)}
%
%\colorlet{circle edge}{blue!50}
%\colorlet{circle area}{blue!20}
%
%\tikzset{filled/.style={fill=circle area, draw=circle edge, thick},
%    outline/.style={draw=circle edge, thick}}
%    
%\begin{tikzpicture}[scale=0.8]
%\draw[outline] \espacio node[above] {$E$};
%    \draw[filled] \firstcircle node {$A$}
%                  \secondcircle node {$B$};
%    \node[anchor=south] at (1,1.3) {$A \cup B$};
%\end{tikzpicture}

%
%\end{document}
	\item $A\cap B=\lbrace4,6\rbrace$\\
	%%\documentclass{standalone}
%\usepackage[utf8]{inputenc}
%\usepackage{graphicx}
%\usepackage{xcolor}
%\usepackage{pgf,tikz}
%\usepackage{mathrsfs}
%\usetikzlibrary{shapes, calc, shapes, arrows, math, babel, positioning}
%
%\begin{document}

\begin{tikzpicture}[scale=0.8]
	\draw[draw=blue!50, thick] (-2,-2) rectangle (4,2) node[above] {$E$};
    \begin{scope}
        \clip (0,0) circle (1.5cm);
        \fill[fill=blue!20, draw=blue!50, thick] (0:2cm) circle (1.5cm);
    \end{scope}
    \draw[draw=blue!50, thick] (0,0) circle (1.5cm) node[left] {$2$};
    \draw[draw=blue!50, thick] (0:2cm) circle (1.5cm) node[right] {$5$};
    %\node[anchor=south] at (current bounding box.north) {$A \cap B$};
    \node[anchor=south] at (1,1.3) {$A \cap B$};
    \node[anchor=south] at (1,0.25) {$4$};
    \node[anchor=south] at (1,-0.75) {$6$};
\end{tikzpicture}



% Original:
%\def\firstcircle{(0,0) circle (1.5cm)}
%\def\secondcircle{(0:2cm) circle (1.5cm)}
%\def\espacio{(-2,-2) rectangle (4,2)}
%
%\colorlet{circle edge}{blue!50}
%\colorlet{circle area}{blue!20}
%
%\tikzset{filled/.style={fill=circle area, draw=circle edge, thick},
%    outline/.style={draw=circle edge, thick}}
%    
%\begin{tikzpicture}[scale=0.8]
%\draw[outline] \espacio node[above] {$E$};
%    \draw[filled] \firstcircle node {$A$}
%                  \secondcircle node {$B$};
%    \node[anchor=south] at (1,1.3) {$A \cup B$};
%\end{tikzpicture}

%
%\end{document}

	\item $A\cup C=\lbrace1,2,3,4,5,6\rbrace=E$
	\item $A\cap C=\varnothing$
\end{itemize}

\end{frame}


\begin{frame}{Compatibilidad de sucesos}
\begin{block}{}
Se dice que dos sucesos son \textbf{incompatibles} cuando su intersección es el conjunto vacío. En caso contrario se dice que son \textbf{compatibles}.
\end{block}

Tomamos como experimento el resultado de lanzar un dado, y los sucesos: \\
\begin{tabular}{l}
$A=\lbrace que\ salga\ par\rbrace=\lbrace2,4,6\rbrace$ \\
$B=\lbrace que\ sea\ mayor\ que\ 3\rbrace=\lbrace4,5,6\rbrace$ \\
$C=\lbrace que\ salga\ impar\rbrace=\lbrace1,3,5\rbrace$
\end{tabular}

$A$ y $B$ son compatibles y $A$ y $C$ incompatibles.

\end{frame}

\begin{frame}{Regla de Laplace}
\begin{block}{}
 La probabilidad de un suceso de un experimento regular viene determinada por la \textbf{Regla de Laplace}:
$$P(A)=\dfrac{Casos\ favorables}{Casos\ posibles} $$
\end{block}

Al lanzar un dado, los casos posibles son 6 ($\lbrace1,2,3,4,5,6\rbrace$):\\
\begin{tabular}{l}
La probabilidad de sacar un 3: $\lbrace3\rbrace\to \dfrac{1}{6}$\\
La probabilidad de sacar par: $\lbrace2,4,6\rbrace\to\dfrac{3}{6}$ \\
La probabilidad de sacar más de 4: $\lbrace5,6\rbrace\to\dfrac{2}{6}$
\end{tabular}

\end{frame}

\begin{frame}{Propiedades de la probabilidad} La probabilidad de un experimento regular cumple las siguientes propiedades:
\begin{itemize}
\item $0 \leq P(A) \leq 1$ 
\item $P(E) = 1$ y $P(\varnothing) = 0$
\item $P(A) + P(\overline A) = 1$
\item $P(A \cup B) = P(A) + P(B) - P(A \cap B)$. Si $A$ y $B$ son incompatibles: $P(A \cup B) = P(A) + P(B)$
\end{itemize}
\end{frame}


\begin{frame}{Probabilidad condicionada}
\begin{block}{}
$$P(A|B)=\dfrac{P(A\cap B)}{P(B)}$$
\end{block}
Despejando:
$$P(A\cap B) = P(A|B)\cdot P(B)$$

\begin{itemize}
\item Se dice que dos sucesos son \textbf{independientes} cuando la probabilidad de cada uno no depende del resultado del otro. 

$$A\ y \ B\ son \ independientes \Longleftrightarrow P(B|A)=P(B)$$

\end{itemize}
\end{frame}

\begin{frame}{Ejemplo sin remplazamiento}
En una urna hay tres bolas blancas y dos negras. Se extraen dos bolas \textbf{sin} reemplazamiento:
%\documentclass{standalone}
%\usepackage[utf8]{inputenc}
%\usepackage{graphicx}
%\usepackage{xcolor}
%\usepackage{pgf,tikz}
%\usepackage{mathrsfs}
%\usetikzlibrary{shapes, calc, shapes, arrows, math, babel, positioning}
%
%\begin{document}

\tikzstyle{bag} = [text width=4em, text centered]
\tikzstyle{end} = [circle, minimum width=3pt,fill, inner sep=0pt]
\tikzstyle{level 1} = [level distance=3.5cm, sibling distance=3.5cm]
\tikzstyle{level 2} = [level distance=3.5cm, sibling distance=2cm]

\begin{tikzpicture}[grow=right, sloped, scale=0.7]
\node[bag] {$3_B, 2_N$}
    child {
        node[bag] {$3_B, 1_N$}        
            child {
                node[end, label=right:
                    {$P(N_1\cap N_2)=\frac{2}{5}\cdot\frac{1}{4}$}] {}
                edge from parent
                node[above] {$N$}
                node[below]  {$1/4$}
            }
            child {
                node[end, label=right:
                    {$P(N_1\cap B_2)=\frac{2}{5}\cdot\frac{3}{4}$}] {}
                edge from parent
                node[above] {$B$}
                node[below] {$3/4$}
            }
            edge from parent 
            node[above] {$N$}
            node[below]  {$2/5$}
    }
    child {
        node[bag] {$2_B, 2_N$}        
        child {
                node[end, label=right:
                    {$P(B_1\cap N_2)=\frac{3}{5}\cdot\frac{2}{4}$}] {}
                edge from parent
                node[above] {$N$}
                node[below]  {$2/4$}
            }
            child {
                node[end, label=right:
                    {$P(B_1\cap B_2)=\frac{3}{5}\cdot\frac{2}{4}$}] {}
                edge from parent
                node[above] {$B$}
                node[below]  {$2/4$}
            }
        edge from parent         
            node[above] {$B$}
            node[below]  {$3/5$}
    };
\end{tikzpicture}

%
%\end{document}
\begin{itemize}
\item Probabilidad de que sean del mismo color: \\
$P((B_1\cap B_2)\cup (N_1\cap N_2))=\frac{3}{10}+\frac{1}{10}=\frac{2}{5}$
\end{itemize}

\end{frame}

\begin{frame}{Ejemplo con remplazamiento}
En una urna hay tres bolas blancas y dos negras. Se extraen dos bolas \textbf{con} reemplazamiento:
%\documentclass{standalone}
%\usepackage[utf8]{inputenc}
%\usepackage{graphicx}
%\usepackage{xcolor}
%\usepackage{pgf,tikz}
%\usepackage{mathrsfs}
%\usetikzlibrary{shapes, calc, shapes, arrows, math, babel, positioning}
%
%\begin{document}

\tikzstyle{bag} = [text width=4em, text centered]
\tikzstyle{end} = [circle, minimum width=3pt,fill, inner sep=0pt]
\tikzstyle{level 1} = [level distance=3.5cm, sibling distance=3.5cm]
\tikzstyle{level 2} = [level distance=3.5cm, sibling distance=2cm]

\begin{tikzpicture}[grow=right, sloped, scale=0.7]
\node[bag] {$3_B, 2_N$}
    child {
        node[bag] {$3_B, 2_N$}        
            child {
                node[end, label=right:
                    {$P(N_1\cap N_2)=\frac{2}{5}\cdot\frac{2}{5}$}] {}
                edge from parent
                node[above] {$N$}
                node[below]  {$2/5$}
            }
            child {
                node[end, label=right:
                    {$P(N_1\cap B_2)=\frac{2}{5}\cdot\frac{3}{5}$}] {}
                edge from parent
                node[above] {$B$}
                node[below] {$3/5$}
            }
            edge from parent 
            node[above] {$N$}
            node[below]  {$2/5$}
    }
    child {
        node[bag] {$3_B, 2_N$}        
        child {
                node[end, label=right:
                    {$P(B_1\cap N_2)=\frac{3}{5}\cdot\frac{2}{5}$}] {}
                edge from parent
                node[above] {$N$}
                node[below]  {$2/5$}
            }
            child {
                node[end, label=right:
                    {$P(B_1\cap B_2)=\frac{3}{5}\cdot\frac{3}{5}$}] {}
                edge from parent
                node[above] {$B$}
                node[below]  {$3/5$}
            }
        edge from parent         
            node[above] {$B$}
            node[below]  {$3/5$}
    };
\end{tikzpicture}

%\end{document}
\begin{itemize}
\item Probabilidad de que sean del mismo color: \\
$P((B_1\cap B_2)\cup (N_1\cap N_2))=\frac{9}{25}+\frac{4}{25}=\frac{13}{25}$
\end{itemize}

\end{frame}



\begin{frame}{Teorema de la probabilidad total}
Si $A_1$, $A_2$, ..., $A_n$   son sucesos incompatibles dos a dos y cuya unión es todo el espacio muestral, entonces la probabilidad de cualquier otro suceso $B$ es:

$$P(B)=\sum_{i=1}^n P(A_i)\cdot  P(B|A_i) $$
\end{frame}

\begin{frame}{Ejemplo de probabilidad total}
En una urna en la que hay tres bolas blancas y dos negras. Si se extraen dos bolas \textbf{sin} reemplazamiento:
%\documentclass{standalone}
%\usepackage[utf8]{inputenc}
%\usepackage{graphicx}
%\usepackage{xcolor}
%\usepackage{pgf,tikz}
%\usepackage{mathrsfs}
%\usetikzlibrary{shapes, calc, shapes, arrows, math, babel, positioning}
%
%\begin{document}

\tikzstyle{bag} = [text width=4em, text centered]
\tikzstyle{end} = [circle, minimum width=3pt,fill, inner sep=0pt]
\tikzstyle{level 1} = [level distance=3.5cm, sibling distance=3.5cm]
\tikzstyle{level 2} = [level distance=3.5cm, sibling distance=2cm]
\begin{tikzpicture}[grow=right, sloped, scale=0.7]
\node[bag] {$3_B, 2_N$}
    child {
        node[bag] {$3_B, 1_N$}        
            child {
                node[end, label=right:
                    {$P(N_1\cap N_2)=\frac{2}{5}\cdot\frac{1}{4}$}] {}
                edge from parent
                node[above] {$N$}
                node[below]  {$1/4$}
            }
            child {
                node[end, label=right:
                    {$P(N_1\cap B_2)=\frac{2}{5}\cdot\frac{3}{4}$}] {}
                edge from parent
                node[above] {$B$}
                node[below] {$3/4$}
            }
            edge from parent 
            node[above] {$N$}
            node[below]  {$2/5$}
    }
    child {
        node[bag] {$2_B, 2_N$}        
        child {
                node[end, label=right:
                    {$P(B_1\cap N_2)=\frac{3}{5}\cdot\frac{2}{4}$}] {}
                edge from parent
                node[above] {$N$}
                node[below]  {$2/4$}
            }
            child {
                node[end, label=right:
                    {$P(B_1\cap B_2)=\frac{3}{5}\cdot\frac{2}{4}$}] {}
                edge from parent
                node[above] {$B$}
                node[below]  {$2/4$}
            }
        edge from parent         
            node[above] {$B$}
            node[below]  {$3/5$}
    };
\end{tikzpicture}

%\end{document}

$$P(B_2)=P(B_1)\cdot P(B_2|B_1) + P(N_1)\cdot P(B_2|N_1)
= \frac{3}{5}\cdot\frac{2}{4} + \frac{2}{5}\cdot\frac{3}{4}$$
\end{frame}


\begin{frame}{Teorema de Bayes}
Si $A_1$, $A_2$, ..., $A_n$   son sucesos incompatibles dos a dos y cuya unión es todo el espacio muestral, y $B$ otro suceso cualquiera:

$$P(A_i|B)=\dfrac{P(A_i \cap B)}{\sum_{i=1}^n P(A_i)\cdot  P(B|A_i)} $$

\end{frame}

\begin{frame}{Ejemplo de Bayes}
En una urna en la que hay tres bolas blancas y dos negras. Si se extraen dos bolas\textbf{sin} reemplazamiento:
%\documentclass{standalone}
%\usepackage[utf8]{inputenc}
%\usepackage{graphicx}
%\usepackage{xcolor}
%\usepackage{pgf,tikz}
%\usepackage{mathrsfs}
%\usetikzlibrary{shapes, calc, shapes, arrows, math, babel, positioning}
%
%\begin{document}

\tikzstyle{bag} = [text width=4em, text centered]
\tikzstyle{end} = [circle, minimum width=3pt,fill, inner sep=0pt]
\tikzstyle{level 1} = [level distance=3.5cm, sibling distance=3.5cm]
\tikzstyle{level 2} = [level distance=3.5cm, sibling distance=2cm]
\begin{tikzpicture}[grow=right, sloped, scale=0.7]
\node[bag] {$3_B, 2_N$}
    child {
        node[bag] {$3_B, 1_N$}        
            child {
                node[end, label=right:
                    {$P(N_1\cap N_2)=\frac{2}{5}\cdot\frac{1}{4}$}] {}
                edge from parent
                node[above] {$N$}
                node[below]  {$1/4$}
            }
            child {
                node[end, label=right:
                    {$P(N_1\cap B_2)=\frac{2}{5}\cdot\frac{3}{4}$}] {}
                edge from parent
                node[above] {$B$}
                node[below] {$3/4$}
            }
            edge from parent 
            node[above] {$N$}
            node[below]  {$2/5$}
    }
    child {
        node[bag] {$2_B, 2_N$}        
        child {
                node[end, label=right:
                    {$P(B_1\cap N_2)=\frac{3}{5}\cdot\frac{2}{4}$}] {}
                edge from parent
                node[above] {$N$}
                node[below]  {$2/4$}
            }
            child {
                node[end, label=right:
                    {$P(B_1\cap B_2)=\frac{3}{5}\cdot\frac{2}{4}$}] {}
                edge from parent
                node[above] {$B$}
                node[below]  {$2/4$}
            }
        edge from parent         
            node[above] {$B$}
            node[below]  {$3/5$}
    };
\end{tikzpicture}

%\end{document}

$$P(B_1|B_2)=\dfrac{P(B_1 \cap B_2)}{P(B_1)\cdot  P(B_2|B_1)+P(N_1)\cdot  P(N_2|B_1)}=\dfrac{\dfrac{3}{5}\cdot\dfrac{2}{4}}{\dfrac{3}{5}\cdot\dfrac{2}{4} + \dfrac{2}{5}\cdot\dfrac{3}{4}}$$
\end{frame}



\end{document}