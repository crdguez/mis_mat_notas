\documentclass[11pt]{beamer}
%\documentclass[11pt,handout]{beamer}
\usetheme{Boadilla}
%\usetheme{metropolis}

\usecolortheme{crane}


\usepackage[utf8]{inputenc}
\usepackage{amsmath, mathtools}
\usepackage{amsfonts}
\usepackage{amssymb}

\usepackage{hyperref}
\usepackage{graphicx}
\usepackage[spanish]{babel}
\graphicspath{{./img/}}

\usepackage{pgf,tikz}

\usetikzlibrary{shapes, calc, shapes, arrows, math, babel, positioning}
\newcommand{\degre}{\ensuremath{^\circ}}
\usepackage{pgf,tikz,pgfplots}
\pgfplotsset{compat=1.15}
\usepackage{mathrsfs}
\usetikzlibrary{arrows}

%\author{}
\title{Sistemas de ecuaciones lineales}

\setbeamercovered{transparent} 
\setbeamertemplate{navigation symbols}{} 
\logo{} 
%\institute{} 
\date{} 
\author{Dep. de Matemáticas}
%\subject{} 
\titlegraphic{\includegraphics[width=0.2\columnwidth]{header_right}}
\begin{document}

\begin{frame}
\titlepage
\end{frame}

%\begin{frame}
%\tableofcontents
%\end{frame}

\begin{frame}{Notación matricial}
Dado un sistema de ecuaciones: 

$${\displaystyle \left\{{\begin{matrix}a_{1,1}x_{1}+a_{1,2}x_{2}+\cdots +a_{1,n}x_{n}&=&b_{1}\\a_{2,1}x_{1}+a_{2,2}x_{2}+\cdots +a_{2,n}x_{n}&=&b_{2}\\\vdots & \vdots & \vdots \\a_{m,1}x_{1}+a_{m,2}x_{2}+\cdots +a_{m,n}x_{n}&=&b_{m}\end{matrix}}\right.}$$

Podemos construir dos matrices: \\

${\displaystyle A={\begin{pmatrix}a_{1,1}&\cdots &a_{1,n}\\\vdots &\ddots &\vdots \\a_{m,1}&\cdots &a_{m,n}\end{pmatrix}}}$ y 
${\displaystyle A^{*}=\left({\begin{matrix}a_{1,1}&\cdots &a_{1,n}\\\vdots &\ddots &\vdots \\a_{m,1}&\cdots &a_{m,n}\end{matrix}}\right|\left.{\begin{matrix}b_{1}\\\vdots \\b_{m}\end{matrix}}\right)}$

A las que llamaremos \textbf{matriz de coeficientes} y \textbf{matriz ampliada} respectivamente.
\end{frame}

\begin{frame}{Notación matricial}
Si llamamos:
${\displaystyle b={\begin{pmatrix}b_{1}\\ \vdots \\b_{m}\end{pmatrix}}}$ y ${\displaystyle x={\begin{pmatrix}x_{1}\\ \vdots \\x_{m}\end{pmatrix}}}$


    \begin{block}{}
    Podemos escribir el sistema en notación matricial:
    \begin{center}
    $A \cdot x = b$
    \end{center}
    \end{block}

   

\end{frame}

\begin{frame}{Teorema de Rouché-Frobenius}

    Podemos determinar cuántas soluciones tiene un sistema de $n$ ecuaciones lineales con $n$ incógnitas a partir del rango de $A$ y $A^*$
    \begin{block}{}
    \begin{itemize}[<+->]
    \item Si $rg(A)=rg(A^*) \Rightarrow$ El sistema es compatible
    \begin{itemize}
        \item Si $rg(A)=rg(A^*)=n \Rightarrow$ El sistema es \textbf{compatible determinado}
        \item Si $rg(A)=rg(A^*)=k<n \Rightarrow$ El sistema es \textbf{compatible indeterminado}
        
    \end{itemize}
    \item Si $rg(A)\neq rg(A^*) \Rightarrow$ El sistema es \textbf{incompatible}
    \end{itemize}
    \end{block}

\end{frame}


\begin{frame}{Ejemplo de Sistema Incompatible}
    Discutir, y si es posible resolver:
$$\left\{ \begin{matrix}5 x - 2 y + z = -1 \\ - 2 x + y - 3 z = 4 \\ 3 x - y - 2 z = 0 \\ \end{matrix}\right.$$

\textbf{Observación:} Es el ejercicio 1.ii del segundo bloque de ejercicios de la página 74

\end{frame}



\begin{frame}{Ejemplo de Sistema Incompatible}
 \begin{itemize}[<+->]
    \item $rango(A) =2$: \\  $A=\left(\begin{matrix}5 & -2 & 1\\-2 & 1 & -3\\3 & -1 & -2\end{matrix}\right) \begin{array}{c}
     \thicksim   \\
     f_3:f_3-f_2-f_1 \\
 \end{array}
 \left(\begin{matrix}5 & -2 & 1\\-2 & 1 & -3\\0 & 0 & 0\end{matrix}\right)$ \\ y $\left|\begin{matrix}5 & -2 \\-2 & 1 \\ \end{matrix}\right| = 5-4=1\neq0 $ 
    \item $rango(A^*) =3$: \\
    $A^*=\left(\begin{matrix}5 & -2 & 1 & -1\\-2 & 1 & -3 & 4\\3 & -1 & -2 & 0\end{matrix}\right)\begin{array}{c}
     \thicksim   \\
    \end{array}
    \left(\begin{matrix}5 & -2 & 1 & -1\\-2 & 1 & -3 & 4\\0 & 0 & 0 & -3\end{matrix}\right)$ y  
    $\left|\begin{matrix}-2 & 1 & -1 \\1 & 3 & 4 \\ 0&0&-3 \end{matrix}\right| = -3\cdot   
    \left|\begin{matrix}-2 & 1\\1 & -3 \\ \end{matrix}\right|=-3\cdot 5=-15\neq0$
    
    \item $rg(A)=2 \neq rg(A^*)=3 \xRightarrow{\text{R-F}} S.I. $
\end{itemize}
\end{frame}

\begin{frame}{Ejemplo de Sistema Incompatible}
\textbf{Observación:} Si hubiésemos seguido el método de Gauss llegaríamos a:

$A^*=\left(\begin{matrix}5 & -2 & 1 & -1\\-2 & 1 & -3 & 4\\3 & -1 & -2 & 0\end{matrix}\right)\begin{array}{c}
     \thicksim   \\
     
    \end{array}
    \left(\begin{matrix}5 & -2 & 1 & -1\\0 & \frac{1}{5} & - \frac{13}{5} & \frac{18}{5}\\0 & 0 & 0 & -3\end{matrix}\right)$
    
    donde la última fila representa la ecuación $0\cdot z = -3$ que no tiene solución.
\end{frame}

\begin{frame}{Ejemplo de Sistema Compatible Determinado}
\textbf{Discutir, y si es posible resolver:
$$\left\{ \begin{matrix}y + z = 1 \\ x + y - 3 z = 8 \\ x - 2 y = -1 \\ \end{matrix}\right.$$}

\textbf{Observación:} Es el ejercicio 3.ii del segundo bloque de ejercicios de la página 74

\end{frame}


\begin{frame}{Ejemplo de Compatible Determinado. Discusión}
\begin{itemize}[<+->]
    \item $rango(A) =3$: \\  $A=\left(\begin{matrix}0 & 1 & 1\\1 & 1 & -3\\1 & -2 & 0\end{matrix}\right)$ \\ y $\left|\begin{matrix}0 & 1 & 1\\1 & 1 & -3\\1 & -2 & 0\end{matrix}\right| = 0-2-3-1-0-0=-6\neq0 $ 
    \item $rango(A^*) =3$ ya que $\left|A\right|$ es un menor de orden 3 de $A^*$ no nulo. 
    
    \item $rg(A)=3=rg(A^*)=3 \xRightarrow{\text{R-F}} S.C.D. $
\end{itemize}
\end{frame}

\begin{frame}{Ejemplo de Compatible Determinado. Resolución}

 Puesto que el rango es máximo $\Rightarrow \exists A^{-1}$, y por lo tanto podemos resolver el sistema por Gauss, por el método de la matriz inversa, o por la regla de Cramer:
\begin{itemize}[<+->]
    \item Por Gauss:
    Haciendo las transformaciones $f_1 \rightleftharpoons f_2 , \ f_3\leftharpoondown f_3-f_1, \  f_3\leftharpoondown f_3+3f_2$:\\
    $A^*=\left(\begin{matrix}0 & 1 & 1 & 1\\1 & 1 & -3 & 8\\1 & -2 & 0 & -1\end{matrix}\right)\thicksim
    \left(\begin{matrix}1 & 1 & -3 & 8\\0 & 1 & 1 & 1\\0 & 0 & 6 & -6\end{matrix}\right)$
    \begin{itemize}
        \item Fila 3: $6z = -6 \to z = -1$
        \item Fila 2: $y+z=1 \to y = 1+1 \to y=2$
        \item Fila 1: $x+y-3z=8 \to x=8-2-3 \to x=3$
    \end{itemize}
\end{itemize}
\end{frame}


\begin{frame}{Ejemplo de Compatible Determinado. Resolución}

\begin{itemize}[<+->]
    \item Por el método del matriz inversa: $A\cdot x=b \Rightarrow x=A^{-1}\cdot b$
    
        \begin{itemize}[<+->]
            \item $Adj(A^t)=\left(\begin{matrix}-6 & -2 & -4\\-3 & -1 & 1\\-3 & 1 & -1\end{matrix}\right)$
            \item $det(A)=\left|\begin{matrix}0 & 1 & 1\\1 & 1 & -3\\1 & -2 & 0\end{matrix}\right|=-6$
            \item $A^{-1}=\frac{\left(\begin{matrix}-6 & -2 & -4\\-3 & -1 & 1\\-3 & 1 & -1\end{matrix}\right)}{-6}=\left(\begin{matrix}1 & \frac{1}{3} & \frac{2}{3}\\\frac{1}{2} & \frac{1}{6} & - \frac{1}{6}\\\frac{1}{2} & - \frac{1}{6} & \frac{1}{6}\end{matrix}\right)$
        \end{itemize}
        \pause
        Operando tenemos que: \\
        $\left(\begin{matrix}x\\y\\z\end{matrix}\right)=\left(\begin{matrix}1 & \frac{1}{3} & \frac{2}{3}\\\frac{1}{2} & \frac{1}{6} & - \frac{1}{6}\\\frac{1}{2} & - \frac{1}{6} & \frac{1}{6}\end{matrix}\right) \cdot \left(\begin{matrix}1\\8\\-1\end{matrix}\right)=\left(\begin{matrix}3\\2\\-1\end{matrix}\right)$
        
\end{itemize}
\end{frame}

\begin{frame}{Ejemplo de Compatible Determinado. Resolución}

\begin{itemize}[<+->]
    \item Por Cramer: 
    \begin{itemize}[<+->]
        \item $x=\frac{\left|\begin{matrix}1 & 1 & 1\\8 & 1 & -3\\-1 & -2 & 0\end{matrix}\right|}{-6}=\frac{-18}{-6}=3$
        \item $y=\frac{\left|\begin{matrix}0 & 1 & 1\\1 & 8 & -3\\1 & -1 & 0\end{matrix}\right|}{-6}=\frac{-12}{-6}=2$
        \item $z=\frac{\left|\begin{matrix}0 & 1 & 1\\1 & 1 & 8\\1 & -2 & -1\end{matrix}\right|}{-6}=\frac{6}{-6}=-1$
    \end{itemize}
\end{itemize}
\end{frame}

\begin{frame}{Ejemplo de Compatible Indeterminado}
\textbf{    Discutir, y si es posible resolver:\\
$$\left\{ \begin{matrix}2 x - y + 3 z = -3 \\ x + y + z = 0 \\ 2 x + 5 y + z = 3 \\ \end{matrix}\right.$$}

\textbf{Observación:} Es el ejercicio 2.i del primer bloque de ejercicios de la página 74
\end{frame}

\begin{frame}{Ejemplo de Compatible Indeterminado. Discusión}
\begin{itemize}[<+->]
    \item $rango(A) =2$: \\ Transformamos a una matriz escalonada \\
    $A=\left(\begin{matrix}2 & -1 & 3\\1 & 1 & 1\\2 & 5 & 1\end{matrix}\right) \begin{array}{c}
     \thicksim   \\
     \end{array}
 \left(\begin{matrix}2 & -1 & 3\\0 & \frac{3}{2} & - \frac{1}{2}\\0 & 0 & 0\end{matrix}\right) $, luego hay dos filas distintas de todo ceros
    \item $rango(A^*) =2$: \\
    $A^*=\left(\begin{matrix}2 & -1 & 3 & -3\\1 & 1 & 1 & 0\\2 & 5 & 1 & 3\end{matrix}\right)\begin{array}{c}
     \thicksim   \\
    \end{array}
    \left(\begin{matrix}2 & -1 & 3 & -3\\0 & \frac{3}{2} & - \frac{1}{2} & \frac{3}{2}\\0 & 0 & 0 & 0\end{matrix}\right)$, de nuevo hay dos filas con elementos no nulos 
    \item $rg(A)=2 = rg(A^*)=2 < 3 \xRightarrow{\text{R-F}} S.C.I. $
\end{itemize}
    
\end{frame}

\begin{frame}{Ejemplo de Compatible Indeterminado. Resolución}
    Al ser de rango $2 < 3$ $\Rightarrow \nexists A^{-1} \to$ No se puede resolver ni por Cramer ni por matriz inversa.
Por Gauss se llega a la matriz escalonada: \\ $A^* \thicksim \left(\begin{matrix}2 & -1 & 3 & -3\\0 & \frac{3}{2} & - \frac{1}{2} & \frac{3}{2}\\0 & 0 & 0 & 0\end{matrix}\right)$:

\begin{itemize}[<+->]
    \item Fila 3: $0z=0 \to \infty  \ soluciones \to z = \lambda$
    \item Fila 2: $3y-z=3 \to y= 1+\frac{\lambda}{3}$
    \item Fila 1: $2x-y+3z=-3 \to x= -1 - \frac{4\lambda}{3} $
\end{itemize}
\end{frame}





\end{document}