\usepackage{pdflscape}
%\usepackage{lscape}
%\usepackage[absolute]{textpos}

\usepackage[spanish]{babel}
%\usepackage[T1]{fontenc}
\usepackage{tikz} 			% Use the calendar.sty style

\usepackage{translator}	% German Month and Day names
\usepackage{fancyhdr}		% header and footer
\usepackage{fix-cm}		% Large year in header

\usepackage[headheight = 2cm, margin=.5cm,
  top = 3.2cm, nofoot]{geometry}
%  \usepackage[landscape, headheight = 2cm, margin=.5cm, top = 3.2cm, nofoot]{geometry}
\usetikzlibrary{calc}
\usetikzlibrary{calendar}
%%%<
\usepackage{verbatim}
\usepackage[tightpage]{preview}
\PreviewEnvironment{tikzpicture}
\setlength\PreviewBorder{5pt}%
%%%>
%\begin{comment}
%:Title: A calender for doublesided DIN-A4
%Tags: Calendar library;Calendars
%:Author: Robert Krause
%
%An example how the calendar package can be used to provide
%an doublesided calendar for the whole year.
%\end{comment}

%\renewcommand*\familydefault{\sfdefault}

% Names of Holidays are inserted by employing this macro
\def\termin#1#2{
  \node [anchor=north west, text width= 3.4cm] at
    ($(cal-#1.north west)+(3em, 0em)$) {\tiny{#2}};
}


\fancypagestyle{style1}{
\fancyhf{}
\renewcommand{\headrulewidth}{0.0pt}
\setlength{\headheight}{0.ex}
}


\fancypagestyle{style2}{
\renewcommand{\headrulewidth}{0.0pt}
\setlength{\headheight}{6.9ex}
\setlength{\footskip}{0.ex}
\fancyfoot{}
\chead{
  \fontsize{30}{36}\selectfont\textbf{\year/\nyear}
  \Large\textbf{Grupo: \rule{3cm}{.5pt}}\hfill
}
}
