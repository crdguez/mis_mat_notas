\documentclass[10pt,a4paper,spanish]{book}
\usepackage[utf8]{inputenc}
\usepackage{amsmath}
\usepackage{amsfonts}
\usepackage{amssymb}
%\usepackage{calc}
\usepackage{pgf}

\usepackage{tikz}
\usetikzlibrary{patterns}
\usepackage{xcolor}
\usepackage{pgffor}
\usepackage[cm]{fullpage}

\def\stdpat       {std}        % Standard, ten squares per inch (s.p.i.)
\def\stdeightpat  {stdeight}   % Standard, eight s.p.i.
\def\majminpat    {majmin}     % Eight s.p.i. w/ major grid ever 1/2 inch
% Indica si std, stdeight, majmin
%\def\usepat{majmin} 

\definecolor{plum}{rgb}{0.36078, 0.20784, 0.4}
\definecolor{chameleon}{rgb}{0.30588, 0.60392, 0.023529}
\definecolor{cornflower}{rgb}{0.12549, 0.29020, 0.52941}
\definecolor{scarlet}{rgb}{0.8, 0, 0}
\definecolor{brick}{rgb}{0.64314, 0, 0}
\definecolor{sunrise}{rgb}{0.80784, 0.36078, 0}

\colorlet{minorcolor}{cornflower!35}
\colorlet{majorcolor}{cornflower!60}

\usepackage{color, colortbl}

%\usetikzlibrary[topaths]

\usepackage{pdflscape}
%\usepackage{lscape}
%\usepackage[absolute]{textpos}

\usepackage[spanish]{babel}
%\usepackage[T1]{fontenc}
\usepackage{tikz} 			% Use the calendar.sty style

\usepackage{translator}	% German Month and Day names
\usepackage{fancyhdr}		% header and footer
\usepackage{fix-cm}		% Large year in header

\usepackage[headheight = 2cm, margin=.5cm,
  top = 3.2cm, nofoot]{geometry}
%  \usepackage[landscape, headheight = 2cm, margin=.5cm, top = 3.2cm, nofoot]{geometry}
\usetikzlibrary{calc}
\usetikzlibrary{calendar}
%%%<
\usepackage{verbatim}
\usepackage[tightpage]{preview}
\PreviewEnvironment{tikzpicture}
\setlength\PreviewBorder{5pt}%
%%%>
%\begin{comment}
%:Title: A calender for doublesided DIN-A4
%Tags: Calendar library;Calendars
%:Author: Robert Krause
%
%An example how the calendar package can be used to provide
%an doublesided calendar for the whole year.
%\end{comment}

%\renewcommand*\familydefault{\sfdefault}

% Names of Holidays are inserted by employing this macro
\def\termin#1#2{
  \node [anchor=north west, text width= 3.4cm] at
    ($(cal-#1.north west)+(3em, 0em)$) {\tiny{#2}};
}


\fancypagestyle{style1}{
\fancyhf{}
\renewcommand{\headrulewidth}{0.0pt}
\setlength{\headheight}{0.ex}
}


\fancypagestyle{style2}{
\renewcommand{\headrulewidth}{0.0pt}
\setlength{\headheight}{6.9ex}
\setlength{\footskip}{0.ex}
\fancyfoot{}
\chead{
  \fontsize{30}{36}\selectfont\textbf{\year/\nyear}
  \Large\textbf{Grupo: \rule{3cm}{.5pt}}\hfill
}
}


% User defined
%\def\year{2019}
%\def\nyear{2020}

\pgfmathsetmacro{\year}{2019}
\pgfmathsetmacro{\nyear}{int(\year + 1)}
\def\autor{Carlos Rodríguez Jaso}
\def\departamento{Matemáticas}
\def\instituto{IES Pedro Cerrada (Utebo)}



\title{\Huge \textbf{Cuaderno de Seguimiento}    \\\bigskip  \huge Departamento de \departamento
%\footnote{\url{}}
\\ \bigskip
{
	\newcount\mycount
	\begin{tikzpicture}[transform shape]
	  %the multiplication with floats is not possible. Thus I split the loop in two.
	  \foreach \number in {1,...,8}{
	      % Computer angle:
	        \mycount=\number
	        \advance\mycount by -1
	  \multiply\mycount by 45
	        \advance\mycount by 0
	      \node[draw,circle,inner sep=0.25cm] (N-\number) at (\the\mycount:5.4cm) {};
	    }
	  \foreach \number in {9,...,16}{
	      % Computer angle:
	        \mycount=\number
	        \advance\mycount by -1
	  \multiply\mycount by 45
	        \advance\mycount by 22.5
	      \node[draw,circle,inner sep=0.25cm] (N-\number) at (\the\mycount:5.4cm) {};
	    }
	  \foreach \number in {1,...,15}{
	        \mycount=\number
	        \advance\mycount by 1
	  \foreach \numbera in {\the\mycount,...,16}{
	    \path (N-\number) edge[->,bend right=3] (N-\numbera)  edge[<-,bend
	      left=3] (N-\numbera);
	  }
	}
	\end{tikzpicture}
}
}
% Author
\author{\textsc{\textbf{\autor}}
%\thanks{\url{http://www.iespedrocerrada.org/}}}
\thanks{\instituto}}
\date{}

\begin{document}

\maketitle
\pagestyle{empty}
\newgeometry{left=.5cm,bottom=.5cm}
\begin{landscape}
\begin{center}
\textbf{\huge{Calendario \year/\nyear}}\\
\begin{tikzpicture}[every day/.style={anchor = north}]
\calendar[
  dates=\year-09-01 to \nyear-02-29,
  name=cal,
  day yshift = 3em,
  day code=
  {
  \node[name=\pgfcalendarsuggestedname,every day,shape=rectangle,
  minimum height= .53cm, text width = 4.4cm, draw = gray]{\tikzdaytext};
  \draw (-1.8cm, -.1ex) node[anchor = west]{\footnotesize%
    \pgfcalendarweekdayshortname{\pgfcalendarcurrentweekday}};
  },
  execute before day scope=
  {
    \ifdate{day of month=1}
    {
    % Shift right
    \pgftransformxshift{4.8cm}
    % Print month name
    \draw (0,0)node [shape=rectangle, minimum height= .53cm,
      text width = 4.4cm, fill = red, text= white, draw = red, text centered]
      {\textbf{\pgfcalendarmonthname{\pgfcalendarcurrentmonth}}};
    }{}
    \ifdate{workday}
{
  % normal days are white
  \tikzset{every day/.style={fill=white}}
  % Vacation  gray background
  \ifdate{between=\year-09-01 and \year-09-12}{%
    \tikzset{every day/.style={fill=gray!30}}}{}
  \ifdate{between=\year-12-21 and \nyear-01-06}{%
    \tikzset{every day/.style={fill=gray!30}}}{}
  \ifdate{equals=10-10}{\tikzset{every day/.style={fill=gray!30}}}{}
  \ifdate{equals=10-11}{\tikzset{every day/.style={fill=gray!30}}}{}
  \ifdate{equals=11-01}{\tikzset{every day/.style={fill=gray!30}}}{}
  \ifdate{equals=05-01}{\tikzset{every day/.style={fill=gray!30}}}{}

}{}
% Saturdays and half holidays (Christma's and New year's eve)
\ifdate{Saturday}{\tikzset{every day/.style={fill=red!10}}}{}
% Sundays and full holidays
\ifdate{Sunday}{\tikzset{every day/.style={fill=red!20}}}{}

  },
 execute at begin day scope=
  {
    % each day is shifted down according to the day of month
    \pgftransformyshift{-.53*\pgfcalendarcurrentday cm}
  }
];

%% No lo uso
% Print name of Holidays
%\termin{\year-11-01}{Todos los santos}
%\termin{\year-01-06}{Heilige Drei Könige}
%\termin{2013-03-29}{Karfreitag}
%\termin{2013-03-31}{Ostersonntag}
%\termin{2013-04-01}{Ostermontag}
%\termin{\year-05-01}{Tag der Arbeit}
%\termin{2013-05-09}{Christi Himmelfahrt}
%\termin{2013-05-19}{Pfingstsonntag}
%\termin{2013-05-20}{Pfingstmontag}
%\termin{2013-05-30}{Fronleichnam}

\end{tikzpicture}


% Repeat the whole thing for the second page
\pagebreak

\textbf{\huge{Calendario \year/\nyear}}\\
\begin{tikzpicture}[every day/.style={anchor = north}]
\calendar[
  dates=\nyear-03-01 to \nyear-08-31,
  name=cal,
  day yshift = 3em,
  day code=
  {
  \node[name=\pgfcalendarsuggestedname,every day,shape=rectangle,
  minimum height= .53cm, text width = 4.4cm, draw = gray]{\tikzdaytext};
  \draw (-1.8cm, -.1ex) node[anchor = west]{\footnotesize%
    \pgfcalendarweekdayshortname{\pgfcalendarcurrentweekday}};
  },
  execute before day scope=
  {
    \ifdate{day of month=1}
    {
    % Shift right
    \pgftransformxshift{4.8cm}
    % Print month name
    \draw (0,0)node [shape=rectangle, minimum height= .53cm,
      text width = 4.4cm, fill = red, text= white, draw = red, text centered]
      {\textbf{\pgfcalendarmonthname{\pgfcalendarcurrentmonth}}};
    }{}
    \ifdate{workday}
{
  % normal days are white
  \tikzset{every day/.style={fill=white}}
  % Vacation  gray background
  \ifdate{between=\year-09-01 and \year-09-12}{%
    \tikzset{every day/.style={fill=gray!30}}}{}
  \ifdate{between=\year-12-21 and \nyear-01-06}{%
    \tikzset{every day/.style={fill=gray!30}}}{}
  \ifdate{equals=10-10}{\tikzset{every day/.style={fill=gray!30}}}{}
  \ifdate{equals=10-11}{\tikzset{every day/.style={fill=gray!30}}}{}
  \ifdate{equals=11-01}{\tikzset{every day/.style={fill=gray!30}}}{}
  \ifdate{equals=05-01}{\tikzset{every day/.style={fill=gray!30}}}{}

}{}
% Saturdays and half holidays (Christma's and New year's eve)
\ifdate{Saturday}{\tikzset{every day/.style={fill=red!10}}}{}
% Sundays and full holidays
\ifdate{Sunday}{\tikzset{every day/.style={fill=red!20}}}{}

  },
 execute at begin day scope=
  {
    % each day is shifted down according to the day of month
    \pgftransformyshift{-.53*\pgfcalendarcurrentday cm}
  }
];

%% No lo uso
% Print name of Holidays
%\termin{\year-11-01}{Todos los santos}
%\termin{\year-01-06}{Heilige Drei Könige}
%\termin{2013-03-29}{Karfreitag}
%\termin{2013-03-31}{Ostersonntag}
%\termin{2013-04-01}{Ostermontag}
%\termin{\year-05-01}{Tag der Arbeit}
%\termin{2013-05-09}{Christi Himmelfahrt}
%\termin{2013-05-19}{Pfingstsonntag}
%\termin{2013-05-20}{Pfingstmontag}
%\termin{2013-05-30}{Fronleichnam}

\end{tikzpicture}

\end{center}

\end{landscape}
\restoregeometry

\pagestyle{style2}
\begin{center}
\begin{tikzpicture}[every day/.style={anchor = north}]
\calendar[
  dates=\year-09-01 to \year-11-30,
  name=cal,
  day yshift = 3em,
  day code=
  {
    \node[name=\pgfcalendarsuggestedname,every day,shape=rectangle,
    minimum height= .8cm, text width = 6.3cm, draw = gray]{ \scriptsize{\tikzdaytext}};
    \draw (-3.cm, -.0ex) node[anchor = west]{\scriptsize%
      \pgfcalendarweekdayshortname{\pgfcalendarcurrentweekday}};
  },
  execute before day scope=
  {
    \ifdate{day of month=1}
    {
      % Shift right
      \pgftransformxshift{6.6cm}
      % Print month name
      \draw (0,0)node [shape=rectangle, minimum height= .7cm,
        text width = 6.3cm, fill = red, text= white, draw = red, text centered]
        {\textbf{\pgfcalendarmonthname{\pgfcalendarcurrentmonth}}};
    }{}
    \ifdate{workday}
{
  % normal days are white
  \tikzset{every day/.style={fill=white}}
  % Vacation  gray background
  \ifdate{between=\year-09-01 and \year-09-12}{%
    \tikzset{every day/.style={fill=gray!30}}}{}
  \ifdate{between=\year-12-21 and \nyear-01-06}{%
    \tikzset{every day/.style={fill=gray!30}}}{}
  \ifdate{equals=10-10}{\tikzset{every day/.style={fill=gray!30}}}{}
  \ifdate{equals=10-11}{\tikzset{every day/.style={fill=gray!30}}}{}
  \ifdate{equals=11-01}{\tikzset{every day/.style={fill=gray!30}}}{}
  \ifdate{equals=05-01}{\tikzset{every day/.style={fill=gray!30}}}{}

}{}
% Saturdays and half holidays (Christma's and New year's eve)
\ifdate{Saturday}{\tikzset{every day/.style={fill=red!10}}}{}
% Sundays and full holidays
\ifdate{Sunday}{\tikzset{every day/.style={fill=red!20}}}{}

  },
 execute at begin day scope=
  {
    % each day is shifted down according to the day of month
    \pgftransformyshift{-.8*\pgfcalendarcurrentday cm}
  }
];

%% No lo uso
% Print name of Holidays
%\termin{\year-11-01}{Todos los santos}
%\termin{\year-01-06}{Heilige Drei Könige}
%\termin{2013-03-29}{Karfreitag}
%\termin{2013-03-31}{Ostersonntag}
%\termin{2013-04-01}{Ostermontag}
%\termin{\year-05-01}{Tag der Arbeit}
%\termin{2013-05-09}{Christi Himmelfahrt}
%\termin{2013-05-19}{Pfingstsonntag}
%\termin{2013-05-20}{Pfingstmontag}
%\termin{2013-05-30}{Fronleichnam}

\end{tikzpicture}

% Repeat the whole thing for the second page
\pagebreak

\begin{tikzpicture}[every day/.style={anchor = north}]
\calendar[
  dates=\year-12-01 to \nyear-02-29,
  name=cal,
  day yshift = 3em,
  day code=
  {
    \node[name=\pgfcalendarsuggestedname,every day,shape=rectangle,
    minimum height= .8cm, text width = 6.3cm, draw = gray]{ \scriptsize{\tikzdaytext}};
    \draw (-3.cm, -.0ex) node[anchor = west]{\scriptsize%
      \pgfcalendarweekdayshortname{\pgfcalendarcurrentweekday}};
  },
  execute before day scope=
  {
    \ifdate{day of month=1}
    {
      % Shift right
      \pgftransformxshift{6.6cm}
      % Print month name
      \draw (0,0)node [shape=rectangle, minimum height= .7cm,
        text width = 6.3cm, fill = red, text= white, draw = red, text centered]
        {\textbf{\pgfcalendarmonthname{\pgfcalendarcurrentmonth}}};
    }{}
    \ifdate{workday}
{
  % normal days are white
  \tikzset{every day/.style={fill=white}}
  % Vacation  gray background
  \ifdate{between=\year-09-01 and \year-09-12}{%
    \tikzset{every day/.style={fill=gray!30}}}{}
  \ifdate{between=\year-12-21 and \nyear-01-06}{%
    \tikzset{every day/.style={fill=gray!30}}}{}
  \ifdate{equals=10-10}{\tikzset{every day/.style={fill=gray!30}}}{}
  \ifdate{equals=10-11}{\tikzset{every day/.style={fill=gray!30}}}{}
  \ifdate{equals=11-01}{\tikzset{every day/.style={fill=gray!30}}}{}
  \ifdate{equals=05-01}{\tikzset{every day/.style={fill=gray!30}}}{}

}{}
% Saturdays and half holidays (Christma's and New year's eve)
\ifdate{Saturday}{\tikzset{every day/.style={fill=red!10}}}{}
% Sundays and full holidays
\ifdate{Sunday}{\tikzset{every day/.style={fill=red!20}}}{}

  },
 execute at begin day scope=
  {
    % each day is shifted down according to the day of month
    \pgftransformyshift{-.8*\pgfcalendarcurrentday cm}
  }
];

%% No lo uso
% Print name of Holidays
%\termin{\year-11-01}{Todos los santos}
%\termin{\year-01-06}{Heilige Drei Könige}
%\termin{2013-03-29}{Karfreitag}
%\termin{2013-03-31}{Ostersonntag}
%\termin{2013-04-01}{Ostermontag}
%\termin{\year-05-01}{Tag der Arbeit}
%\termin{2013-05-09}{Christi Himmelfahrt}
%\termin{2013-05-19}{Pfingstsonntag}
%\termin{2013-05-20}{Pfingstmontag}
%\termin{2013-05-30}{Fronleichnam}

\end{tikzpicture}

% Repeat the whole thing for the second page
\pagebreak

\begin{tikzpicture}[every day/.style={anchor = north}]
\calendar[
  dates=\nyear-03-01 to \nyear-05-31,
  name=cal,
  day yshift = 3em,
  day code=
  {
    \node[name=\pgfcalendarsuggestedname,every day,shape=rectangle,
    minimum height= .8cm, text width = 6.3cm, draw = gray]{ \scriptsize{\tikzdaytext}};
    \draw (-3.cm, -.0ex) node[anchor = west]{\scriptsize%
      \pgfcalendarweekdayshortname{\pgfcalendarcurrentweekday}};
  },
  execute before day scope=
  {
    \ifdate{day of month=1}
    {
      % Shift right
      \pgftransformxshift{6.6cm}
      % Print month name
      \draw (0,0)node [shape=rectangle, minimum height= .7cm,
        text width = 6.3cm, fill = red, text= white, draw = red, text centered]
        {\textbf{\pgfcalendarmonthname{\pgfcalendarcurrentmonth}}};
    }{}
    \ifdate{workday}
{
  % normal days are white
  \tikzset{every day/.style={fill=white}}
  % Vacation  gray background
  \ifdate{between=\year-09-01 and \year-09-12}{%
    \tikzset{every day/.style={fill=gray!30}}}{}
  \ifdate{between=\year-12-21 and \nyear-01-06}{%
    \tikzset{every day/.style={fill=gray!30}}}{}
  \ifdate{equals=10-10}{\tikzset{every day/.style={fill=gray!30}}}{}
  \ifdate{equals=10-11}{\tikzset{every day/.style={fill=gray!30}}}{}
  \ifdate{equals=11-01}{\tikzset{every day/.style={fill=gray!30}}}{}
  \ifdate{equals=05-01}{\tikzset{every day/.style={fill=gray!30}}}{}

}{}
% Saturdays and half holidays (Christma's and New year's eve)
\ifdate{Saturday}{\tikzset{every day/.style={fill=red!10}}}{}
% Sundays and full holidays
\ifdate{Sunday}{\tikzset{every day/.style={fill=red!20}}}{}

  },
 execute at begin day scope=
  {
    % each day is shifted down according to the day of month
    \pgftransformyshift{-.8*\pgfcalendarcurrentday cm}
  }
];

%% No lo uso
% Print name of Holidays
%\termin{\year-11-01}{Todos los santos}
%\termin{\year-01-06}{Heilige Drei Könige}
%\termin{2013-03-29}{Karfreitag}
%\termin{2013-03-31}{Ostersonntag}
%\termin{2013-04-01}{Ostermontag}
%\termin{\year-05-01}{Tag der Arbeit}
%\termin{2013-05-09}{Christi Himmelfahrt}
%\termin{2013-05-19}{Pfingstsonntag}
%\termin{2013-05-20}{Pfingstmontag}
%\termin{2013-05-30}{Fronleichnam}

\end{tikzpicture}

% Repeat the whole thing for the second page
\pagebreak

\begin{tikzpicture}[every day/.style={anchor = north}]
\calendar[
  dates=\nyear-06-01 to \nyear-08-31,
  name=cal,
  day yshift = 3em,
  day code=
  {
    \node[name=\pgfcalendarsuggestedname,every day,shape=rectangle,
    minimum height= .8cm, text width = 6.3cm, draw = gray]{ \scriptsize{\tikzdaytext}};
    \draw (-3.cm, -.0ex) node[anchor = west]{\scriptsize%
      \pgfcalendarweekdayshortname{\pgfcalendarcurrentweekday}};
  },
  execute before day scope=
  {
    \ifdate{day of month=1}
    {
      % Shift right
      \pgftransformxshift{6.6cm}
      % Print month name
      \draw (0,0)node [shape=rectangle, minimum height= .7cm,
        text width = 6.3cm, fill = red, text= white, draw = red, text centered]
        {\textbf{\pgfcalendarmonthname{\pgfcalendarcurrentmonth}}};
    }{}
    \ifdate{workday}
{
  % normal days are white
  \tikzset{every day/.style={fill=white}}
  % Vacation  gray background
  \ifdate{between=\year-09-01 and \year-09-12}{%
    \tikzset{every day/.style={fill=gray!30}}}{}
  \ifdate{between=\year-12-21 and \nyear-01-06}{%
    \tikzset{every day/.style={fill=gray!30}}}{}
  \ifdate{equals=10-10}{\tikzset{every day/.style={fill=gray!30}}}{}
  \ifdate{equals=10-11}{\tikzset{every day/.style={fill=gray!30}}}{}
  \ifdate{equals=11-01}{\tikzset{every day/.style={fill=gray!30}}}{}
  \ifdate{equals=05-01}{\tikzset{every day/.style={fill=gray!30}}}{}

}{}
% Saturdays and half holidays (Christma's and New year's eve)
\ifdate{Saturday}{\tikzset{every day/.style={fill=red!10}}}{}
% Sundays and full holidays
\ifdate{Sunday}{\tikzset{every day/.style={fill=red!20}}}{}

  },
 execute at begin day scope=
  {
    % each day is shifted down according to the day of month
    \pgftransformyshift{-.8*\pgfcalendarcurrentday cm}
  }
];

%% No lo uso
% Print name of Holidays
%\termin{\year-11-01}{Todos los santos}
%\termin{\year-01-06}{Heilige Drei Könige}
%\termin{2013-03-29}{Karfreitag}
%\termin{2013-03-31}{Ostersonntag}
%\termin{2013-04-01}{Ostermontag}
%\termin{\year-05-01}{Tag der Arbeit}
%\termin{2013-05-09}{Christi Himmelfahrt}
%\termin{2013-05-19}{Pfingstsonntag}
%\termin{2013-05-20}{Pfingstmontag}
%\termin{2013-05-30}{Fronleichnam}

\end{tikzpicture}

% Repeat the whole thing for the second page
\pagebreak



\end{center}


\pagebreak

\pagestyle{empty}
%\definecolor{LightCyan}{rgb}{0.88,1,1}
\definecolor{Gray}{gray}{0.95}

%\def\rot{\rotatebox}

% \usepackage{array} is required
\begin{tabular}{>{\raggedleft\arraybackslash}p{2.2cm}l}
\textbf{Grupo:} & \rule{3cm}{.5pt}  \\
\textbf{Evaluación:} & \rule{3cm}{.5pt} \\
\textbf{Tutor:} & \rule{3cm}{.5pt} \\
\end{tabular}

\begin{table}[th]
\centering
\begin{tabular}{r p{6cm}|p{0.4cm}|p{0.4cm}|p{0.4cm}|p{0.4cm}|p{0.4cm}|p{0.4cm}|p{0.4cm}|p{0.4cm}|p{0.4cm}|p{0.4cm}|p{0.4cm}|p{0.4cm}|p{0.4cm}|p{0.4cm}|}
%\hline
&  &  &  &  &  &  &  &  &  &  &  &  &  &  &\\ [4cm]
\hline
\hline
\rowcolor{Gray}
 1- &  &  &  &  &  &  &  &  &  &  &  &  &  &  &\\
  &  &  &  &  &  &  &  &  &  &  &  &  &  &  &\\
 \rowcolor{Gray}
  &  &  &  &  &  &  &  &  &  &  &  &  &  &  &\\
  &  &  &  &  &  &  &  &  &  &  &  &  &  &  &\\
\hline
\end{tabular}
\end{table}

\newpage



% Indica si std, stdeight, majmin
\def\usepat{stdeight}
\foreach \n in {1,...,3}{ \pagebreak

% Indica si std, stdeight, majmin
%\def\usepat{stdeight}



\begin{tikzpicture}[remember picture, overlay]

  % Change "very thin" to "thin" if the lines are too thin.
  \tikzset{
    minorgrid/.style={minorcolor, thin},
    majorgrid/.style={majorcolor, thin},
  }

\ifx\usepat\stdpat
% Draw a grid with 10 squares per inch.
\draw[style=minorgrid, shift={(current page.south west)},shift={(0.2in,0.2in)}] (0,0) coordinate (a) grid
[step=0.1in] (20.0cm,27.95cm) coordinate (b);

% Draw a frame around the grid.
\draw[style=majorgrid] (a) rectangle (b);
\fi

\ifx\usepat\stdeightpat
% Draw a grid with 10 squares per inch.
\draw[style=minorgrid, shift={(current page.south west)},shift={(0.1875in,0.1875in)}] (0,0) coordinate (a)
grid [step=0.125in] (20.0cm,27.95cm) coordinate (b);

% Draw a frame around the grid.
\draw[style=majorgrid] (a) rectangle (b);
\fi

\ifx\usepat\majminpat
% Draw a grid with 10 squares per inch.
\draw[style=minorgrid, shift={(current page.south west)},shift={(0.225in,0.25in)}] (0,0) coordinate (a) grid [step=0.125in] (20.0cm,27.95cm) coordinate (b);

\draw[style=majorgrid, shift={(current page.south west)},shift={(0.225in,0.25in)}] (0,0) coordinate (a) grid [step=0.5in] (20.0cm,27.95cm) coordinate (b);

% Draw a frame around the grid.
\draw[style=majorgrid] (a) rectangle (b);
\fi

\end{tikzpicture}
}

% Indica si std, stdeight, majmin
\def\usepat{majmin}
\foreach \n in {1,...,3}{ \pagebreak

% Indica si std, stdeight, majmin
%\def\usepat{stdeight}



\begin{tikzpicture}[remember picture, overlay]

  % Change "very thin" to "thin" if the lines are too thin.
  \tikzset{
    minorgrid/.style={minorcolor, thin},
    majorgrid/.style={majorcolor, thin},
  }

\ifx\usepat\stdpat
% Draw a grid with 10 squares per inch.
\draw[style=minorgrid, shift={(current page.south west)},shift={(0.2in,0.2in)}] (0,0) coordinate (a) grid
[step=0.1in] (20.0cm,27.95cm) coordinate (b);

% Draw a frame around the grid.
\draw[style=majorgrid] (a) rectangle (b);
\fi

\ifx\usepat\stdeightpat
% Draw a grid with 10 squares per inch.
\draw[style=minorgrid, shift={(current page.south west)},shift={(0.1875in,0.1875in)}] (0,0) coordinate (a)
grid [step=0.125in] (20.0cm,27.95cm) coordinate (b);

% Draw a frame around the grid.
\draw[style=majorgrid] (a) rectangle (b);
\fi

\ifx\usepat\majminpat
% Draw a grid with 10 squares per inch.
\draw[style=minorgrid, shift={(current page.south west)},shift={(0.225in,0.25in)}] (0,0) coordinate (a) grid [step=0.125in] (20.0cm,27.95cm) coordinate (b);

\draw[style=majorgrid, shift={(current page.south west)},shift={(0.225in,0.25in)}] (0,0) coordinate (a) grid [step=0.5in] (20.0cm,27.95cm) coordinate (b);

% Draw a frame around the grid.
\draw[style=majorgrid] (a) rectangle (b);
\fi

\end{tikzpicture}
}

\end{document}
