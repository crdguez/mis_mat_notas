\hypertarget{distribuciones-de-probabilidad}{%
\section{Distribuciones de
probabilidad}\label{distribuciones-de-probabilidad}}

\hypertarget{finalidad}{%
\paragraph{Finalidad :}\label{finalidad}}

Abstraer matemáticamente un tipo de experimento aleatorio. Y, con ello,
poder estimar de manera teórica lo que sucedería de manera experimental
mediante una estadística

\hypertarget{cuxf3mo}{%
\paragraph{¿Cómo?:}\label{cuxf3mo}}

Mediante variables aleatorias y distribuciones de probabilidad asociadas
a esas variables

\hypertarget{variables-aleatorias}{%
\subsection{Variables aleatorias}\label{variables-aleatorias}}

Una variable aleatoria es una función que a cada suceso elemental de un
espacio muestral le asigna un número. Para hacer referencia a las
variables se usan las letras: \(X\), \(Y\), ...

\hypertarget{ejemplo}{%
\paragraph{Ejemplo:}\label{ejemplo}}

Sea el experimento aleatorio ``lanzar un dado'': El espacio muestral lo
componen las 6 caras del dado. Podemos asignar la variable \(X\) que a
cada cara le asocia el número que represente su cara.

\begin{longtable}[]{@{}lcl@{}}
\toprule
\endhead
& \(X\) &\tabularnewline
Suceso & & \(x_i\)\tabularnewline
Cara 1 & \(\rightarrow\) & 1\tabularnewline
Cara 2 & \(\rightarrow\) & 2\tabularnewline
Cara 3 & \(\rightarrow\) & 3\tabularnewline
Cara 4 & \(\rightarrow\) & 4\tabularnewline
Cara 5 & \(\rightarrow\) & 5\tabularnewline
Cara 6 & \(\rightarrow\) & 6\tabularnewline
\bottomrule
\end{longtable}

\hypertarget{ejemplo-1}{%
\paragraph{Ejemplo:}\label{ejemplo-1}}

Sea el experimento compuesto lanzar dos monedas. Podemos asignar la
variable aleatoria: \[Y=\left\lbrace Número \ de \ caras \right\rbrace\]

\begin{longtable}[]{@{}lcl@{}}
\toprule
\endhead
& \(Y\) &\tabularnewline
Suceso & & \(y_i\)\tabularnewline
C,C & \(\rightarrow\) & 2\tabularnewline
C,X & \(\rightarrow\) & 1\tabularnewline
X,C & \(\rightarrow\) & 1\tabularnewline
X,X & \(\rightarrow\) & 0\tabularnewline
\bottomrule
\end{longtable}

\hypertarget{tipos-de-variables-aleatorias}{%
\subsubsection{Tipos de variables
aleatorias}\label{tipos-de-variables-aleatorias}}

\begin{description}
\item[Discretas:]
Toman un número finito o numerable de valores
\item[Continuas:]
Toman valores en un rango continuo
\end{description}

\hypertarget{ejemplo-de-variables-discreta}{%
\paragraph{Ejemplo de variables
discreta:}\label{ejemplo-de-variables-discreta}}

Sea la \emph{X =``El número de caras al lanzar dos dados''}. Los valores
posibles son 0, 1 o 2 (que es un conjunto finito de datos, en concreto 3
datos)

\hypertarget{ejemplo-de-variables-continua}{%
\paragraph{Ejemplo de variables
continua:}\label{ejemplo-de-variables-continua}}

\emph{X = ``Distancia al centro de la diana medida desde la posición en
que cae un dardo lanzado por un tirador experto''} . En este caso la
variable puede tomar cualquier valor en el rango entre 0 y el radio de
la diana

\hypertarget{distribuciones-de-probabilidad-1}{%
\subsection{Distribuciones de
probabilidad}\label{distribuciones-de-probabilidad-1}}

Llamaremos Distribución de probabilidad a la relación entre los valores
de la variable y sus probabilidades.

Estas relaciones se pueden indicar mediante funciones. El tratamiento de
estas funciones es diferente según las variables sean discretas o
continuas.

\hypertarget{ejemplo-2}{%
\paragraph{Ejemplo:}\label{ejemplo-2}}

Sea la variable \emph{X = ``Número obtenido al lanzar una dado''}\\
A cada valor de la variable podemos asignarle su probabilidad:

\begin{longtable}[]{@{}lcl@{}}
\toprule
\endhead
& \(P(X)\) &\tabularnewline
\(x_i\) & & \(P(x_i)\)\tabularnewline
1 & \(\rightarrow\) & \(\tfrac{1}{6}\)\tabularnewline
2 & \(\rightarrow\) & \(\tfrac{1}{6}\)\tabularnewline
3 & \(\rightarrow\) & \(\tfrac{1}{6}\)\tabularnewline
4 & \(\rightarrow\) & \(\tfrac{1}{6}\)\tabularnewline
5 & \(\rightarrow\) & \(\tfrac{1}{6}\)\tabularnewline
6 & \(\rightarrow\) & \(\tfrac{1}{6}\)\tabularnewline
\bottomrule
\end{longtable}
