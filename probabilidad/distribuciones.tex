\section{Distribuciones de probabilidad}

\paragraph{Finalidad :} Abstraer matemáticamente un tipo de experimento aleatorio. Y, con ello, poder estimar de manera teórica lo que sucedería de manera experimental mediante una estadística 

\paragraph{¿Cómo?:} Mediante variables aleatorias y distribuciones de probabilidad asociadas a esas variables

\subsection{Variables aleatorias} Una variable aleatoria es una función que a cada suceso
elemental de un espacio muestral le asigna un número. Para hacer referencia a las variables se usan las letras: $X$, $Y$, ...

\paragraph{Ejemplo:} Sea el experimento aleatorio “lanzar un dado”: El espacio muestral lo componen las 6 caras del dado. Podemos asignar la variable $X$ que a cada cara le asocia el número que represente su cara.

\begin{center}
\begin{tabular}{ccc}
 & $X$ &  \\
Suceso &  &  $x_i$\\ \hline 
Cara 1 & $\rightarrow$ & 1 \\ 
Cara 2 & $\rightarrow$ & 2 \\ 
Cara 3 & $\rightarrow$ & 3 \\ 
Cara 4 & $\rightarrow$ & 4 \\ 
Cara 5 & $\rightarrow$ & 5 \\ 
Cara 6 & $\rightarrow$ & 6 \\ 
\end{tabular} 
\end{center}

\paragraph{Ejemplo:} Sea el experimento compuesto lanzar dos monedas. Podemos asignar la variable aleatoria: $$Y=\left\lbrace Número \ de \ caras \right\rbrace$$

\begin{center}
\begin{tabular}{ccc}
 & $Y$ &  \\
Suceso &  &  $y_i$\\ \hline 
C,C & $\rightarrow$ & 2 \\ 
C,X & $\rightarrow$ & 1 \\ 
X,C & $\rightarrow$ & 1 \\ 
X,X & $\rightarrow$ & 0 \\ 
\end{tabular} 
\end{center}

\subsubsection{Tipos de variables aleatorias}
\begin{description}
\item[Discretas:] Toman un número finito o numerable de valores
\item[Continuas:] Toman valores en un rango continuo
\end{description}

\paragraph{Ejemplo de variables discreta:} Sea la \emph{X =“El número de caras al lanzar dos dados”}. Los valores posibles son 0, 1 o 2 (que es un conjunto finito de datos, en concreto 3 datos) 

\paragraph{Ejemplo de variables continua:} \emph{X = “Distancia al centro de la diana medida desde la posición en que cae un dardo lanzado por un tirador experto” }. En este caso la variable puede tomar cualquier valor en el rango entre 0 y el radio de la diana 

\subsection{Distribuciones de probabilidad}

Llamaremos Distribución de probabilidad a la relación entre los valores de la variable y sus probabilidades.

Estas relaciones se pueden indicar mediante funciones. El tratamiento de estas funciones es diferente según las variables sean discretas o continuas.

\subsubsection{Distribución uniforme discreta}

\paragraph{Ejemplo: } Sea la variable \emph{X = “Número obtenido al lanzar una dado”}\\
A cada valor de la variable podemos asignarle su probabilidad:

%\begin{center}
%\begin{tabular}{ccc}
% & $P(X)$ &  \\
%Suceso &  &  $y_i$ \\ \hline 
%1 & $\rightarrow$ & \frac{1}{6} \\ 
%2 & $\rightarrow$ & \frac{1}{6} \\ 
%3 & $\rightarrow$ & \frac{1}{6} \\ 
%4 & $\rightarrow$ & \frac{1}{6} \\ 
%5 & $\rightarrow$ & \frac{1}{6} \\ 
%6 & $\rightarrow$ & \frac{1}{6} \\ 
%\end{tabular} 
%\end{center}

\begin{center}
\begin{tabular}{ccc}
 & $P(X)$ &  \\
$x_i$ &  &  $P(x_i)$\\ \hline 
1 & $\rightarrow$ & $\tfrac{1}{6}$ \\ 
2 & $\rightarrow$ & $\tfrac{1}{6}$ \\ 
3 & $\rightarrow$ & $\tfrac{1}{6}$ \\ 
4 & $\rightarrow$ & $\tfrac{1}{6}$ \\ 
5 & $\rightarrow$ & $\tfrac{1}{6}$ \\ 
6 & $\rightarrow$ & $\tfrac{1}{6}$ \\ 
\end{tabular} 
\end{center}

Podemos representar la relación anterior mediante una función:


%$$f\colon \begin{array}{>{\displaystyle}l} 
%          X \rightarrow Y \\ 
%          x\mapsto f(x)=\frac{x-1}{2} 
%         \end{array}$$

$$P\colon \begin{array}{>{\displaystyle}l} 
          X \rightarrow \mathbb{R} \\ 
          x_i\rightarrow P(X=x_i)=\frac{1}{n} 
         \end{array}$$
Todas las caras tienen la misma probabilidad: $\frac{1}{n}$, siendo $n$ el número de caras, o tamaño del espacio muestral. 

A este tipo de distribución se le llama \textbf{uniforme discreta}.

\subsubsection{Distribución Binomial}

\paragraph{Ejemplo:} Queremos calcular las probabilidades de que al lanzar 5 monedas, obtengamos tres caras.
Un suceso que cumple el enunciado es:
$$S_1=\left\lbrace C,C,C,X,X \right\rbrace$$
Teniendo en cuenta que lanzar cada moneda son experimentos independientes, la probabilidad de ese suceso será:
%$$P\left(S_1\right)=P\left(C_1\right)\cdot P\left(C_2 C_1 \right)$$
\begin{eqnarray*}
P\left(S_1\right) & = &P\left(C_1\right)\cdot P\left(C_2 | C_1 \right)\cdot ... \cdot P\left(X_5 | C_1 \cap C_2  \cap C_3  \cap X_4   \right)= \\ &  = & P\left(C_1\right)\cdot  P\left(C_2\right) \cdot P\left(C_3\right) \cdot P\left(X_4\right) \cdot P\left(X_5\right)= \\
& = & P\left(C\right)^3\cdot  P\left(X\right)^2
\end{eqnarray*}

Como la probabilidad de que una moneda sea cara es $\frac{1}{2}$ y la de que sea cruz también:

$$P\left(S_1\right)=\left(\frac{1}{2}\right)^3\cdot  \left(\frac{1}{2}\right)^2$$

Ahora bien, habrá tantos sucesos que cumplan el enunciado como combinaciones de 5 elementos tomados de 3 en 3. Por tanto la probabilidad de que salgan 3 caras será:

$$P\left(Salgan \ 3 \ caras\right)=\binom{5}{3}\left(\frac{1}{2}\right)^3\cdot  \left(\frac{1}{2}\right)^2$$

Si asociamos al experimento "lanzar 5 monedas" le asignamos la variable "número de caras obtenidas", podemos determinar la probabilidad mediante la siguiente función.

$$P\colon \begin{array}{>{\displaystyle}l} 
          X \rightarrow \mathbb{R} \\ 
          k\rightarrow P(X=k)=\binom{5}{k}\left(\frac{1}{2}\right)^k\cdot  \left(\frac{1}{2}\right)^{5-k} 
         \end{array}$$

Esto es un ejemplo de distribución binomial de tamaño $5$ y probabilidad $0.5$ .


\paragraph{Generalización de distribución Binomial:} Hablaremos de una distribución binomial cuando:
\begin{itemize}
\item Se parte de un experimento compuesto de varios simples independientes
\item Los experimentos simples son dicotómicos. Es decir, solo puede haber dos sucesos elementales: uno al que llamaremos acierto y otro al que llamaremos fracaso
\item Asociado al experimento compuesto tenemos la variable número de aciertos cuando realizamos el experimento simple un número determinado de veces
\end{itemize}

En la situación anterior, la distribución binomial vendrá determinada por dos parámetros:
\begin{itemize}
\item Parámetro $n$: Número de veces que se realiza el experimento simple 
\item Parámetro $p$: La probabilidad de que ocurra el suceso acierto
\end{itemize}


A este tipo de variable y su distribución de probabilidades se le llama binomial

\paragraph{Ejemplo:} Así en el ejemplo de las monedas, el experimento se compone de 5 lanzamientos de moneda. Si sale cara es acierto y si no fracaso. La variable aleatoria asociada al experimento será el número de caras que salen al lanzar 5 monedas. Esta variable sigue una distribución binomial y por tanto:

En general tendremos una binomial de tamaño n y probabilidad p, cuando el experimento simple se haga n veces y la probabilidad de acierto se p.

La función de probabilidad en este caso nos queda:
 

	











